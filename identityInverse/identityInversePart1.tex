\documentclass[space,nooutcomes]{ximera} 
%\documentclass[space,handout,nooutcomes]{ximera} 

\title{Identities and Inverses, Part 1}

\begin{document}
\begin{abstract}
Identities and inverses arise in lots of mathematical
settings.  What is the same across these settings?  
\end{abstract}
\maketitle


\begin{question}
Note:  Your ``free response'' answers are not checked for accuracy.  To optimize your learning, we recommend you submit your own answer before revealing the hint.  

Complete the following sentence: 

To \wordChoice{\choice[correct]{optimize}\choice{minimize}} my learning, I plan to $\answer[format=string]{submit}$ my own answer \wordChoice{\choice[correct]{before}\choice{after}} revealing the $\answer[format=string]{hint}$.  
\end{question}

\section*{Introduction}
How is an inverse matrix like a multiplicative inverse or an inverse function?  

When using the terms \emph{identity} and \emph{inverse}, it is
important to be clear about the objects (e.g., numbers, matrices,
functions) and operations involved.  Also, because the meaning of
inverse depends upon an identity element, it is good practice to
discuss the identity first.

\begin{definition}
We call $0$ the \emph{additive identity} because when the operation is addition 
$0$ behaves as an identity element in most sets of numbers.  
In other words, $x+0=x$ and $0+x=x$ for all $x$ in the relevant number
set (e.g., integers).
\end{definition}

\begin{question}
Which of the following sets of numbers \textbf{do not} have an additive identity?  (Select all.)
\begin{selectAll}
\choice[correct]{counting numbers}
\choice{whole numbers}
\choice[correct]{odd numbers}
\choice{even numbers}
\choice[correct]{non-zero rational numbers}
\choice{rational numbers}
\choice[correct]{irrational numbers}
\choice{real numbers}
\choice{complex numbers}
\end{selectAll}
\end{question}

In any set of numbers, when you have an additive identity,
you can define the term additive inverse.  

\begin{definition}
Given a number $x$, a number $y$ is said to be the \emph{additive
  inverse} of $x$ if $x+y=0$ and $y+x=0$.  
\end{definition}

If $y$ is the additive inverse of $x$, it follows immediately that
$x$ is the additive inverse of $y$.  Thus, when both of these
conditions hold, we say that $x$ and $y$ are additive inverses of each
other.

\begin{question}
Note that the additive identity is a single element that works for
all elements of the set, whereas ``additive inverse'' is a
relationship between a pair of elements.  Combine both of these ideas
together in a single sentence.  

\begin{solution}
When we $\answer[format=string]{add}$ additive inverses, we get the
$\answer[format=string]{additive identity}$.
\end{solution}
\end{question}

\begin{question}
Are there numbers that are their own additive inverses?  If not,
explain.  If so, name all that you can, and explain how you know you
have them all.

\begin{solution}
A number $x$ that is its own additive inverse must be a solution to
the equation $\answer{x}+\answer{x} = \answer{0}$.  This is equivalent to $\answer{2x}=0$, and the only
solution is $x=\answer{0}$.  Thus, $\answer{0}$ is the only number that is its own
additive inverse.
\end{solution}
\end{question}

\begin{question}
Name some sets of numbers in which some of the elements don't have
additive inverses.
\begin{freeResponse}
\begin{hint}
Possible answers: counting numbers, whole numbers, the set ${1, 3,
  4}$.
\end{hint}
\end{freeResponse}
\end{question}

\begin{question}Define \emph{multiplicative identity} for your favorite set of numbers: 
\begin{solution}
In the real numbers, we call $\answer{1}$ the \emph{multiplicative identity} because when the operation is 
$\answer[format=string]{multiplication}$,  
$\answer{1}$ behaves as an identity element.  
In other words, $x\cdot \answer{1}=\answer{x}$ and $\answer{1}\cdot x=\answer{x}$ for all real numbers $x$.  
\end{solution}
\end{question}

\begin{question}Define \emph{multiplicative inverse} for your favorite set of numbers:   

Given a real number $x$, a real number $y$ is the multiplicative inverse of $x$ if $x\cdot\answer{y}=\answer{1}$ 
and $\answer{y}\cdot x=\answer{1}$.  

\end{question}

\begin{question}Identity elements are sometimes called ``do-nothing elements.''  Use that idea to talk about both the additive identity and the multiplicative identity for real numbers.  
\begin{freeResponse}
\begin{hint}
Given any number $x$, adding $0$, the additive identity, yields $x$, so the identity ``does nothing.''  Similarly, multiplying $x$ by the multiplicative identity yields $x$.  So $1$ is a do-nothing element for multiplication.  
\end{hint}
\end{freeResponse}
\end{question}

\begin{question}
Are there numbers that are their own multiplicative inverses?  If not, explain.  If so, name all that you can, and explain how you know you have them all.
\begin{solution}
A number $x$ that is its own multiplicative inverse would be a solution to the equation $\answer{x}\cdot \answer{x}=\answer{1}$ or $\answer{x^2}=\answer{1}$.  In the real numbers, this equation has exactly $\answer{2}$ solutions.   
\begin{question}
Yes.  Enter those solutions from least to greatest: 

$x = \answer{-1}$ or $x = \answer{1}$.
\end{question}
\end{solution}
\end{question}

\section*{Matrices}
When discussing matrices, there are lots of size restrictions
regarding whether two matrices can be added or multiplied.  To keep
the discussion simple, let's consider only $2\times 2$ matrices.  The
ideas generalize fairly easily to $n\times n$ matrices, where $n$ is a
counting number.  Non-square matrices are more problematic.

When we discuss an \emph{identity matrix} and the \emph{inverse of a
  matrix}, the operation is assumed to be matrix multiplication.  In
other words, the $2\times 2$ identity matrix might be more
appropriately called the \emph{multiplicative identity matrix}.

\begin{question}
What would be the $2\times 2$ additive identity matrix?  Show how you
know.
\begin{solution}
The additive identity matrix is $\begin{bmatrix} \answer{0}&\answer{0} \\ \answer{0}&\answer{0} \end{bmatrix}$.  

\begin{question}
This is true because 
\[
\begin{bmatrix} a&b \\ c&d \end{bmatrix} + \begin{bmatrix} 0&0 \\ 0&0 \end{bmatrix} = 
\begin{bmatrix} a&b \\ c&d \end{bmatrix}
\text{ and }
\begin{bmatrix} 0&0 \\ 0&0 \end{bmatrix} + \begin{bmatrix} a&b \\ c&d \end{bmatrix} = 
\begin{bmatrix} a&b \\ c&d \end{bmatrix}\text{, for all }a, b, c, d.  
\]
Note that if $A = \begin{bmatrix} a&b \\ c&d \end{bmatrix}$ and $Z = \begin{bmatrix} 0&0 \\ 0&0 \end{bmatrix}$, this can be written simply:  
\[
A + \answer{Z} = \answer{A}\text{ and }\answer{Z}+A = \answer{A}.
\]
\end{question}
\end{solution}
\end{question}

\begin{question}
What would be the additive inverse of the matrix 
$\begin{bmatrix} a&b \\ c&d \end{bmatrix}$?  
Show how you know.  
\begin{solution}
The additive inverse of $\begin{bmatrix} a&b \\ c&d \end{bmatrix}$ is 
$\begin{bmatrix} \answer{-a}&\answer{-b} \\ \answer{-c}&\answer{-d} \end{bmatrix}$.  

\begin{question}
This is true because 

$\begin{bmatrix} a&b \\ c&d \end{bmatrix} + \begin{bmatrix} -a&-b \\ -c&-d \end{bmatrix} = 
\begin{bmatrix} 0&0 \\ 0&0 \end{bmatrix}$ 
and
$\begin{bmatrix} -a&-b \\ -c&-d \end{bmatrix} + \begin{bmatrix} a&b \\ c&d \end{bmatrix} = 
\begin{bmatrix} 0&0 \\ 0&0 \end{bmatrix}$ for all $a$, $b$, $c$, $d$. 

Note that if $A$ and $Z$ are as in the previous question, then $-A = \begin{bmatrix} \answer{-a}&\answer{-b} \\ \answer{-c}&\answer{-d} \end{bmatrix}$, and this can be written simply:  
\[
A + \answer{-A} = \answer{Z}\text{ and }\answer{-A} + A = \answer{Z}. 
\]
\end{question}
\end{solution}
\end{question}

\begin{question}
What is the $2\times 2$ (multiplicative) identity matrix?  Show how you know.
\begin{solution}
$I = \begin{bmatrix} \answer{1}&\answer{0} \\ \answer{0}&\answer{1} \end{bmatrix}$ is the multiplicative identity matrix because for any $2\times 2$ matrix $A$, 
\[
A\cdot\answer{I} = \answer{A}\text{ and }\answer{I}\cdot A=\answer{A}.  
\]
\begin{question}
Here is more detail:  

$\begin{bmatrix} a&b \\ c&d \end{bmatrix}  \begin{bmatrix} 1&0 \\ 0&1 \end{bmatrix} = 
\begin{bmatrix} a&b \\ c&d \end{bmatrix}$  
and
$\begin{bmatrix} 1&0 \\ 0&1 \end{bmatrix} \begin{bmatrix} a&b \\ c&d \end{bmatrix} = 
\begin{bmatrix} a&b \\ c&d \end{bmatrix}$  for all $a$, $b$, $c$, $d$.  
\end{question}
\end{solution}
\end{question}

\begin{question}
What does it mean for matrices $A$ and $B$ to be (multiplicative) inverses of each other?  

\begin{solution}
Matrices $A$ and $B$ are inverses of each other if $A\cdot\answer{B}=\answer{I}$ and $\answer{B}\cdot A=\answer{I}$.  
\end{solution}
\end{question}

\subsection*{A Joke.}  After writing two matrices on the board, a professor asks a student, ``Are these matrices inverses?''  The student answers, ``The first one is, and the second one isn't.''  
\begin{question}
Why is this joke funny?  
\begin{freeResponse}
\begin{hint}
The student thinks that ``inverse'' can apply to a single object, but ``inverse'' is a relationship between a pair of objects.  
\end{hint}
\end{freeResponse}
\end{question}

\begin{question}
Are there matrices that are their own (multiplicative) inverses?  If not, explain.  If so, list a few, and describe them all.
\begin{solution}
A matrix $A$ that is its own multiplicative inverse would be a solution to the equation $A\cdot \answer{A}=\answer{I}$ or $\answer{A^2}=I$.  

This equation equation has \wordChoice{\choice{zero}\choice{exactly one}\choice{exactly two}\choice{exactly four}\choice{exactly eight}\choice{thousands of}\choice[correct]{an infinite number of}} solutions.   
\begin{feedback}
Correct:  There are an infinite number of matrices that satisfy this equation!
Here are four convenient ones: 

$I=\begin{bmatrix} 1&0 \\ 0&1 \end{bmatrix}$, $\begin{bmatrix} -1&0 \\ 0&1 \end{bmatrix}$, $\begin{bmatrix} 1&0 \\ 0&-1 \end{bmatrix}$, and $\begin{bmatrix} -1&0 \\ 0&-1 \end{bmatrix}=\answer{-I}$. 
\end{feedback}
\end{solution}
\end{question}

\begin{question}
Describe what ``identity'' and ``inverse'' mean across all of these settings.  
\begin{freeResponse}
\begin{hint}
For any operation, the identity is the ``do-nothing'' element for that operation, and inverses are \text{pairs} of elements that combine (under that operation) to yield the identity element. Together, an element and its inverse combine to ``do nothing'' under that operation.  
\end{hint}
\end{freeResponse}
\end{question}

\end{document}

