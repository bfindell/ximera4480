\documentclass{ximera}
%\documentclass[space,handout,nooutcomes]{ximera}

\title{Identities and Inverses Part 1}

\begin{document}
\begin{abstract}
Identities and inverses arise in lots of different mathematical settings.  In this activity, we aim to see what is the same across these settings.  
\end{abstract}
\maketitle


%
%\begin{description}
%
%\end{description}
When using the terms \emph{identity} and \emph{inverse}, it is important to be clear about the objects (e.g., numbers, matrices, functions) and operations involved.  Also, because the meaning of inverse depends upon an identity element, it is good practice to discuss the identity first.  

We call $0$ the \emph{additive identity} because it is the identity element in most sets of numbers when the operation is addition.  In other words, $x+0=x$ and $0+x=x$ for all $x$ in the relevant number set (e.g., whole numbers, integers, rational numbers, or real numbers). 
\begin{question}
Name some sets of numbers that don't have an additive identity.  
\begin{freeResponse}
Possible answers:  counting numbers, odd numbers, non-zero rational numbers.  
\end{freeResponse}
\end{question}

In your favorite set of numbers, when you have an additive identity, you can define the term additive inverse.  Given a number $x$, a number $y$ in your set of numbers is said to be the \emph{additive inverse} of $x$ if $x+y=0$ and $y+x=0$.  It follows immediately that $x$ is the additive inverse of $y$.  Thus, when both of these conditions hold, we say that $x$ and $y$ are additive inverses of each other.  

Note that the additive identity is a single element that ``works'' for all elements of the set, whereas ``additive inverse'' is a relationship between pairs of elements.  We can put both ideas together as follows:  When we add additive inverses, we get the additive identity.
\begin{question}
Are there numbers that are their own additive inverses?  If not, explain.  If so, name all that you can, and explain how you know you have them all.  
\begin{freeResponse}
A number $x$ that is its own additive inverse must be a solution to the equation $x+x = 0$.  This is equivalent to $2x=0$, and the only solution is $x=0$.  Thus, 0 is the only number that is its own additive inverse.   
\end{freeResponse}
\end{question}

\begin{question}Name some sets of numbers in which some of the elements don't have additive inverses.  
\begin{freeResponse}
Possible answers:  counting numbers, whole numbers, the set ${1, 3, 4}$.  
\end{freeResponse}
\end{question}

\begin{question}Define \emph{multiplicative identity} for your favorite set of numbers.  
\begin{freeResponse}
\end{freeResponse}
\end{question}

\begin{question}Define \emph{multiplicative identity} for your favorite set of numbers.  
\begin{freeResponse}
\end{freeResponse}
\end{question}

\begin{question}Identity elements are sometimes called ``do nothing elements.''  Use that idea to talk about both the additive identity and the multiplicative identity for real numbers.  
\begin{freeResponse}
\end{freeResponse}
\end{question}

\begin{question}
Are there numbers that are their own multiplicative inverses?  If not, explain.  If so, name all that you can, and explain how you know you have them all.
\begin{freeResponse}
A number $x$ that is its own multiplicative inverse would be a solution to the equation $x\dot x=1$ or $x^2=1$.  In the real numbers, this equation has two solutions:  $x = \pm 1$.  
\end{freeResponse}
\end{question}

When discussing matrices, there are lots of size restrictions regarding whether two matrices can be added or multiplied.  To keep the discussion simple, let's consider only $2\times 2$ matrices.  The ideas generalize fairly easily to $n\times n$  matrices, if $n$ is a counting number, but non-square matrices are more problematic.  

When we discuss an \emph{identity matrix} and the \emph{inverse of a matrix}, the operation is assumed to be matrix multiplication.  In other words, the $2\times 2$ identity matrix might be more appropriately called the \emph{multiplicative identity matrix}:  It is really the identity with respect to matrix multiplication.  

\begin{question}
What would be the $2\times 2$ additive identity matrix?  Show how you know.  
\begin{freeResponse}
$\begin{bmatrix} 0&0 \\ 0&0 \end{bmatrix}$
\end{freeResponse}
\end{question}

\begin{question}
What would be the additive inverse of the matrix 
$\begin{bmatrix} a&b \\ c&d \end{bmatrix}$?  Show how you know.  
\begin{freeResponse}
$\begin{bmatrix} -a&-b \\ -c&-d \end{bmatrix}$
\end{freeResponse}
\end{question}

\begin{question}
What is the $2\times 2$ multiplicative identity matrix?  Show how you know.
\begin{freeResponse}
\end{freeResponse}
\end{question}

\begin{question}
What does it mean for matrices $A$ and $B$ to be multiplicative inverses of each other?  
\begin{freeResponse}
\end{freeResponse}
\end{question}

\textbf{A joke.}  After writing two matrices on the board, a professor asks a student, ``Are these matrices inverses?''  The student answers, ``The first one is, and the second one isn't.''  
\begin{question}
Why is this joke funny?  
\begin{freeResponse}
\end{freeResponse}
\end{question}


\end{document}

