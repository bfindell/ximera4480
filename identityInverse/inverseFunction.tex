\documentclass[space,nooutcomes]{ximera} 
%\documentclass[space,handout,nooutcomes]{ximera} 

\title{Inverses of Functions}

\begin{document}
\begin{abstract}
More on inverses of functions.
\end{abstract}
\maketitle

Consider the following problem and ``solution.''

\begin{problem}

Express $x$ in terms of $y$ if $y = \ln x$.  
\begin{solution}
\begin{align*}
y & = \ln x \\
x & = \frac{y}{\ln} \\
x & = \left(\frac{1}{\ln}\right)\left(y\right) \\
x & = \ln^{-1}(y) \\
x & = e^y
\end{align*}
\end{solution}
\end{problem}

\begin{problem}
Express $x$ in terms of $y$ if $y = \ln x$.  
\begin{solution}
Let $f(x) = \ln x$ and $g(x) = e^x$.  Then $g^{-1}(x) = f(x)$ and $f^{-1}(x) = g(x)$.  In other words, $g^{-1} = f$ and $f^{-1}=g$.  

Here is (sort-of) the reasoning:  
\begin{align*}
y & = \ln x \\
y & = f(x) \\
f^{-1}(y) & = f^{-1}(f(x)) \\
f^{-1}(y) & = x \\
x & = f^{-1}(y) \\
x & = g(y) \\
x & = e^y
\end{align*}
\end{solution}

Note 1:  We do not use exponents for function composition for $e^x$ nor for $\ln x$.  

Note 2: Because function composition is not commutative, 
fractional notation should not be used.  Instead, stick to exponents.  

\end{problem}

\end{document}

