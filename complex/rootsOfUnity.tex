\documentclass[space,nooutcomes]{ximera} 

% For preamble materials

\usepackage{pgf,tikz}
\usepackage{mathrsfs}
\usetikzlibrary{arrows}
\usepackage{framed}
\usepackage{amsmath}
\pgfplotsset{compat=1.13}
%\pgfplotsset{compat=1.16}

\usepackage{numprint} % For printing large numbers with commas. 
\npthousandsep{,}


\graphicspath{
  {./}
  {quickQuestions/}
  {ximeraTutorial/}
}


\pdfOnly{\renewenvironment{image}[1][]{\begin{center}}{\end{center}}}

%%% This set of code is all of our user defined commands
\newcommand{\bysame}{\mbox{\rule{3em}{.4pt}}\,}
\newcommand{\N}{\mathbb N}
\newcommand{\C}{\mathbb C}
\newcommand{\W}{\mathbb W}
\newcommand{\Z}{\mathbb Z}
\newcommand{\Q}{\mathbb Q}
\newcommand{\R}{\mathbb R}
\newcommand{\A}{\mathbb A}
\newcommand{\D}{\mathcal D}
\newcommand{\F}{\mathcal F}
\newcommand{\ph}{\varphi}
\newcommand{\ep}{\varepsilon}
\newcommand{\aph}{\alpha}
\newcommand{\QM}{\begin{center}{\huge\textbf{?}}\end{center}}

\renewcommand{\le}{\leqslant}
\renewcommand{\ge}{\geqslant}
\renewcommand{\a}{\wedge}
\renewcommand{\v}{\vee}
\renewcommand{\l}{\ell}
\newcommand{\mat}{\mathsf}
\renewcommand{\vec}{\mathbf}
\renewcommand{\subset}{\subseteq}
\renewcommand{\supset}{\supseteq}
\renewcommand{\emptyset}{\varnothing}
\newcommand{\xto}{\xrightarrow}
\renewcommand{\qedsymbol}{$\blacksquare$}
\newcommand{\bibname}{References and Further Reading}
\renewcommand{\bar}{\protect\overline}
\renewcommand{\hat}{\protect\widehat}
\renewcommand{\tilde}{\widetilde}
\newcommand{\tri}{\triangle}
\newcommand{\minipad}{\vspace{1ex}}
\newcommand{\leftexp}[2]{{\vphantom{#2}}^{#1}{#2}}

%% More user defined commands
\renewcommand{\epsilon}{\varepsilon}
%\renewcommand{\theta}{\vartheta} %% only for kmath
\renewcommand{\l}{\ell}
\renewcommand{\d}{\, d}
\newcommand{\ddx}{\frac{d}{dx}}
\newcommand{\dydx}{\frac{dy}{dx}}


\usepackage{bigstrut}


\newenvironment{sectionOutcomes}{}{}

\usepackage{array}
%\setlength{\extrarowheight}{-.2cm}   % Commented out by Findell to fix table headings.  Was this for typesetting division?  
\newdimen\digitwidth
\settowidth\digitwidth{9}
\def~{\hspace{\digitwidth}}
\def\divrule#1#2{
\noalign{\moveright#1\digitwidth
\vbox{\hrule width#2\digitwidth}}}


\title{Roots of Unity}
\author{Brad Findell}
\begin{document}
\begin{abstract}
Problems exploring roots of unity.   
\end{abstract}
\maketitle


\begin{definition}
Given a positive integer $n$, a solution to the equation $x^n=1$ is called an \textbf{$n^\text{th}$  root of unity}.    

If $x$ is an $n^\text{th}$ root of unity and $x^k\ne 1$ for all integers $k$, $0<k<n$, then $x$ is called a \textbf{primitive $n^\text{th}$ root of unity}.
\end{definition}
 
\begin{problem}
In a previous activity, you found the $4^\text{th}$ roots of unity.  They were the solutions to the equation (in $x$)
\[
\answer{x^4} = 1.  
\]
How many solutions did you find?  $\answer{4}$
\begin{problem}
List the $4^\text{th}$ roots of unity in increasing order of their angles, $\theta$, assuming $0\le \theta<2\pi$: 
\[
\answer{1}, \answer{i}, \answer{-1}\text{, and }\answer{-i}. 
\]
\begin{problem}
For each of the $4^\text{th}$ roots of unity, indicate the smallest positive $k$ for which $x^k = 1$.  
\begin{enumerate}
\item $x=1$, $k=\answer{1}$.  
\item $x=i$, $k=\answer{4}$.
\item $x=-1$, $k=\answer{2}$.  
\item $x=-i$, $k=\answer{4}$.
\end{enumerate}
So how many of these $4^\text{th}$ roots of unity are \textbf{primitive} $4^\text{th}$ roots of unity?  
$\answer{2}$
\begin{problem}
Correct!  Which ones?  $\answer{i}$ and $\answer{-i}$.  
\end{problem}
\end{problem}
\end{problem}
\end{problem}

\begin{problem}
In a previous activity, you found the $6^\text{th}$ roots of unity.  They were the solutions to the equation (in $x$)
\[
\answer{x^6} = 1.  
\]
How many solutions did you find?  $\answer{6}$
\begin{problem}
List the $6^\text{th}$ roots of unity in increasing order of their angles, $\theta$, assuming $0\le \theta<2\pi$: 
\[
\answer{1}, \answer{\frac{1+i\sqrt{3}}{2}}, \answer{\frac{-1+i\sqrt{3}}{2}}, 
\answer{-1}, \answer{\frac{-1-i\sqrt{3}}{2}}\text{, and }\answer{\frac{1-i\sqrt{3}}{2}}. 
\]
\begin{problem}
For each of the $6^\text{th}$ roots of unity, indicate the smallest positive $k$ for which $x^k = 1$.  
\begin{enumerate}
\item $x=1$, $k=\answer{1}$.  
\item $x=\frac{1+i\sqrt{3}}{2}$, $k=\answer{6}$.
\item $x=\frac{-1+i\sqrt{3}}{2}$, $k=\answer{3}$.
\item $x=-1$, $k=\answer{2}$.  
\item $x=\frac{-1-i\sqrt{3}}{2}$, $k=\answer{3}$.
\item $x=\frac{1-i\sqrt{3}}{2}$, $k=\answer{6}$.
\end{enumerate}
So how many of these $6^\text{th}$ roots of unity are \textbf{primitive} $6^\text{th}$ roots of unity?  
$\answer{2}$
\begin{problem}
Correct!  Which ones?  $\answer{\frac{1+i\sqrt{3}}{2}}$ and $\answer{\frac{1-i\sqrt{3}}{2}}$.  
\begin{problem}
Yes!  

And $x=-1$ is a primitive \wordChoice{\choice{first} \choice[correct]{square} \choice{cube} \choice{fourth} \choice{$n^\text{th}$} } root of unity,  
while $x=\frac{-1+i\sqrt{3}}{2}$ and $x=\frac{-1-i\sqrt{3}}{2}$ are primitive \wordChoice{\choice{first} \choice{square} \choice[correct]{cube} \choice{fourth} \choice{$n^\text{th}$} } roots of unity. 
\end{problem}
\end{problem}
\end{problem}
\end{problem}
\end{problem}

\begin{problem}
If $z = a + bi$ (with $b$ not $0$) is a primitive $n^\text{th}$ root of unity, then is the conjugate of $z$ also a primitive $n^\text{th}$ root of unity?  Make a conjecture and prove it.
\vfill 
\end{problem}


\begin{problem}
Can you find all the $n^\text{th}$ roots of unity for $n = 2, 3, 4, 5, 6, 7, 8, 9$, and $10$?  For each $n$, can you determine which of the $n^\text{th}$ roots of unity are \textbf{primitive} $n^\text{th}$ roots of unity?  Be sure that you are able to demonstrate that each root is actually a primitive $n^\text{th}$ root of unity and also that you have found them all.  
\end{problem}


\end{document}
