\documentclass[space,nooutcomes]{ximera} 

% For preamble materials

\usepackage{pgf,tikz}
\usepackage{mathrsfs}
\usetikzlibrary{arrows}
\usepackage{framed}
\usepackage{amsmath}
\pgfplotsset{compat=1.16}

\usepackage{numprint} % For printing large numbers with commas. 
\npthousandsep{,}


\graphicspath{
  {./}
  {quickQuestions/}
  {ximeraTutorial/}
}


\pdfOnly{\renewenvironment{image}[1][]{\begin{center}}{\end{center}}}

%%% This set of code is all of our user defined commands
\newcommand{\bysame}{\mbox{\rule{3em}{.4pt}}\,}
\newcommand{\N}{\mathbb N}
\newcommand{\C}{\mathbb C}
\newcommand{\W}{\mathbb W}
\newcommand{\Z}{\mathbb Z}
\newcommand{\Q}{\mathbb Q}
\newcommand{\R}{\mathbb R}
\newcommand{\A}{\mathbb A}
\newcommand{\D}{\mathcal D}
\newcommand{\F}{\mathcal F}
\newcommand{\ph}{\varphi}
\newcommand{\ep}{\varepsilon}
\newcommand{\aph}{\alpha}
\newcommand{\QM}{\begin{center}{\huge\textbf{?}}\end{center}}

\renewcommand{\le}{\leqslant}
\renewcommand{\ge}{\geqslant}
\renewcommand{\a}{\wedge}
\renewcommand{\v}{\vee}
\renewcommand{\l}{\ell}
\newcommand{\mat}{\mathsf}
\renewcommand{\vec}{\mathbf}
\renewcommand{\subset}{\subseteq}
\renewcommand{\supset}{\supseteq}
\renewcommand{\emptyset}{\varnothing}
\newcommand{\xto}{\xrightarrow}
\renewcommand{\qedsymbol}{$\blacksquare$}
\newcommand{\bibname}{References and Further Reading}
\renewcommand{\bar}{\protect\overline}
\renewcommand{\hat}{\protect\widehat}
\renewcommand{\tilde}{\widetilde}
\newcommand{\tri}{\triangle}
\newcommand{\minipad}{\vspace{1ex}}
\newcommand{\leftexp}[2]{{\vphantom{#2}}^{#1}{#2}}

%% More user defined commands
\renewcommand{\epsilon}{\varepsilon}
%\renewcommand{\theta}{\vartheta} %% only for kmath
\renewcommand{\l}{\ell}
\renewcommand{\d}{\, d}
\newcommand{\ddx}{\frac{d}{dx}}
\newcommand{\dydx}{\frac{dy}{dx}}


\usepackage{bigstrut}


\newenvironment{sectionOutcomes}{}{}

\usepackage{array}
%\setlength{\extrarowheight}{-.2cm}   % Commented out by Findell to fix table headings.  Was this for typesetting division?  
\newdimen\digitwidth
\settowidth\digitwidth{9}
\def~{\hspace{\digitwidth}}
\def\divrule#1#2{
\noalign{\moveright#1\digitwidth
\vbox{\hrule width#2\digitwidth}}}


\title{Fifth Roots of Unity}
\author{Brad Findell}
\begin{document}
\begin{abstract}
A sequence of problems deriving the fifth roots of unity.   
\end{abstract}
\maketitle



\begin{problem}

To find the fifth roots of unity, we need to solve the equation: 
\begin{equation}
x^5 = 1,
\end{equation}
or equivalently, 
\begin{equation} \label{eqA}
\answer{x^5 - 1} = 0.
\end{equation}

\begin{problem}
Clearly $x=1$ is a solution, so $\answer{(x-1)}$ must be a factor.  

\begin{problem}
The left side of (\ref{eqA}) factors as follows: 
\begin{equation}
(x-1)\left(\answer{x^4+x^3+x^2+x+1}\right)=0.
\end{equation}
\end{problem}
\end{problem}
\end{problem}

\begin{problem}
From previous explorations, we know that roots of unity all lie on the $\answer[format=string]{unit circle}$ (two words) in the complex plane. 

If $x$ is an $n^\text{th}$ root of unity, what do we know about these quantities: 
\begin{itemize}
\item $x^{-1}$ 
\item $\overline{x}$
\item $x+x^{-1}$
\end{itemize}

Answers:
\begin{enumerate}
\item $x^{-1}$ is an $n^\text{th}$ root of $\answer[format=string]{unity}$.  
\item $\overline{x} = \answer{x^{-1}}$. 
\item $x+x^{-1}$ is a $\answer[format=string]{real}$ number.  
\end{enumerate}
\begin{problem}
Explanations: 
\begin{enumerate}
\item Because $x^n=1$, it follows that 
\[
\left(x^{-1}\right)^n = \answer{x^{-n}} = \left(x^n\right)^{-1} = 1^{-1} = 1. 
\]
\item Suppose $x=a+bi$, with $a$ and $b$ real numbers. Because $x$ is on the unit circle, $a^2+b^2 = \answer{1}$.  Then 
\[
x^{-1}=\frac{1}{a+bi}=\frac{1}{a+bi}\cdot\frac{a-bi}{\answer{a-bi}}=\frac{a-bi}{\answer{a^2+b^2}}=\frac{a-bi}{1}=\overline{x}.
\]
\item For any complex number, $x+\overline{x}$ is a(n) $\answer[format=string]{real}$ number.  When $x$ is on the unit circle, 
\[
x + \overline{x} = x + x^{-1} = x + \frac{1}{x}\text{ is a(n) }\answer[format=string]{real}\text{ number.}
\]
So in the following problems, we organize the work around $x + \frac{1}{x}$.  
\end{enumerate}
\end{problem}
\end{problem}


\begin{problem}

To solve: 
\begin{equation}
x^4+x^3+x^2+x+1=0,
\end{equation}

divide by $x^2$, and then organize around powers of $x + \frac{1}{x}$: 
\begin{align}
x^2+x+1+\frac{1}{x}+\frac{1}{x^2}&=0, \\
\left(x^2+ \answer{2} + \frac{1}{x^2}\right)+ \left(x+\frac{1}{x}\right)+\answer{-1}&=0. \label{eqB}
\end{align}
\begin{hint}
The expression inside the first parentheses of (\ref{eqB}) needs to be equivalent to 
\[
\left(x+\frac{1}{x}\right)^2.
\]
\end{hint}
\begin{problem}
Now let $y = x + \frac{1}{x}$, and equation (\ref{eqB}) becomes: 
\begin{equation} \label{eqC}
\answer{y^2+y-1}=0.
\end{equation}

Solve equation (\ref{eqC}) for $y$.

How many solutions?  $\answer{2}$ 

\begin{problem}
Enter the solutions to (\ref{eqC}), from least to greatest: 

\begin{prompt}
$y = \answer{\frac{-1-\sqrt{5}}{2}}$, or 

$y = \answer{\frac{-1+\sqrt{5}}{2}}$.
\end{prompt}

\end{problem}
\end{problem}
\end{problem}

\begin{problem}
We defined $y$ as follows: 
\begin{equation} \label{eqD}
y = x + \frac{1}{x}.  
\end{equation}
Solve equation (\ref{eqD}) for $x$ in terms of $y$.  
\begin{hint}
Multiply to clear the fractions, then recognize that equation (\ref{eqD}) is quadratic in $x$.  
% $x^2 - yx + 1 = 0
\end{hint}

How many solutions?  $\answer{2}$ 
\begin{problem}
\begin{prompt}
Enter the solutions to (\ref{eqD}), for $x$ in terms of $y$, as a conjugate pair: 

$x = \answer{\frac{y}{2}} \pm \answer{\frac{\sqrt{y^2-4}}{2}}$.
\end{prompt}

\end{problem}
\end{problem}


\begin{problem}
Now put the previous answers together to yield all $\answer{4}$ complex solutions $x$ to the equation
\begin{equation} \label{eqE}
x^4+x^3+x^2+x+1=0.
\end{equation}
\begin{problem}
Enter the solutions to (\ref{eqE}) as conjugate pairs: 

Quadrants 1 and 4: $\left(\answer{\frac{-1+\sqrt{5}}{4}}\right) 
     \pm \left(\answer{\frac{\sqrt{10+2\sqrt{5}}}{4}}\right)i$;

Quadrants 2 and 3: $\left(\answer{\frac{-1-\sqrt{5}}{4}}\right)
     \pm \left(\answer{\frac{\sqrt{10-2\sqrt{5}}}{4}}\right)i$.

\end{problem}
\end{problem}

\end{document}
