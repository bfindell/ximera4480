\documentclass[space,nooutcomes]{ximera} 

% For preamble materials

\usepackage{pgf,tikz}
\usepackage{mathrsfs}
\usetikzlibrary{arrows}
\usepackage{framed}
\usepackage{amsmath}
\pgfplotsset{compat=1.16}

\usepackage{numprint} % For printing large numbers with commas. 
\npthousandsep{,}


\graphicspath{
  {./}
  {quickQuestions/}
  {ximeraTutorial/}
}


\pdfOnly{\renewenvironment{image}[1][]{\begin{center}}{\end{center}}}

%%% This set of code is all of our user defined commands
\newcommand{\bysame}{\mbox{\rule{3em}{.4pt}}\,}
\newcommand{\N}{\mathbb N}
\newcommand{\C}{\mathbb C}
\newcommand{\W}{\mathbb W}
\newcommand{\Z}{\mathbb Z}
\newcommand{\Q}{\mathbb Q}
\newcommand{\R}{\mathbb R}
\newcommand{\A}{\mathbb A}
\newcommand{\D}{\mathcal D}
\newcommand{\F}{\mathcal F}
\newcommand{\ph}{\varphi}
\newcommand{\ep}{\varepsilon}
\newcommand{\aph}{\alpha}
\newcommand{\QM}{\begin{center}{\huge\textbf{?}}\end{center}}

\renewcommand{\le}{\leqslant}
\renewcommand{\ge}{\geqslant}
\renewcommand{\a}{\wedge}
\renewcommand{\v}{\vee}
\renewcommand{\l}{\ell}
\newcommand{\mat}{\mathsf}
\renewcommand{\vec}{\mathbf}
\renewcommand{\subset}{\subseteq}
\renewcommand{\supset}{\supseteq}
\renewcommand{\emptyset}{\varnothing}
\newcommand{\xto}{\xrightarrow}
\renewcommand{\qedsymbol}{$\blacksquare$}
\newcommand{\bibname}{References and Further Reading}
\renewcommand{\bar}{\protect\overline}
\renewcommand{\hat}{\protect\widehat}
\renewcommand{\tilde}{\widetilde}
\newcommand{\tri}{\triangle}
\newcommand{\minipad}{\vspace{1ex}}
\newcommand{\leftexp}[2]{{\vphantom{#2}}^{#1}{#2}}

%% More user defined commands
\renewcommand{\epsilon}{\varepsilon}
%\renewcommand{\theta}{\vartheta} %% only for kmath
\renewcommand{\l}{\ell}
\renewcommand{\d}{\, d}
\newcommand{\ddx}{\frac{d}{dx}}
\newcommand{\dydx}{\frac{dy}{dx}}


\usepackage{bigstrut}


\newenvironment{sectionOutcomes}{}{}

\usepackage{array}
%\setlength{\extrarowheight}{-.2cm}   % Commented out by Findell to fix table headings.  Was this for typesetting division?  
\newdimen\digitwidth
\settowidth\digitwidth{9}
\def~{\hspace{\digitwidth}}
\def\divrule#1#2{
\noalign{\moveright#1\digitwidth
\vbox{\hrule width#2\digitwidth}}}


\title{Polar Form}
\author{Brad Findell}
\begin{document}
\begin{abstract}
Exploring complex numbers in polar form.   
\end{abstract}
\maketitle



% The notation $\text{cis}\theta$ is a shorthand for $\cos\theta +i\sin\theta$, sometimes used in Precalculus texts. 

%You have seen Euler's formula: $e^{i\theta}=\cos\theta +i\sin\theta$.  


\begin{problem}
Thinking of the complex plane as ordered pairs of real numbers, the number $a+bi$ can be represented in rectangular coordinates as the point $(a,b)$ or as a vector from the origin to $(a,b)$.  
\begin{center}
\definecolor{ududff}{rgb}{0.30196078431372547,0.30196078431372547,1.}
\begin{tikzpicture}[line cap=round,line join=round,>=triangle 45,x=1.0cm,y=1.0cm]
\begin{axis}[
x=1.0cm,y=1.0cm,
axis lines=middle,
xmin=-2.5,
xmax=6.3,
ymin=-2.4,
ymax=3.4,
%xlabel={Re($z$)},
%ylabel={Im($z$)},
xticklabel=\empty,
yticklabel=\empty,]
\clip(-2.8,-2.7) rectangle (6.5,4.);
\draw [line width=1.2pt] (0.,0.)-- (3.48,2.1);
\draw [line width=1.2pt] (3.48,2.1)-- (3.48,0.);
\draw [line width=1.2pt] (0.,0.)-- (3.48,0.);
\draw (5.4,0) node[below] {Re(z)};
\draw (0,2.8) node[left] {Im(z)};
\begin{scriptsize}
\draw [fill=ududff] (3.48,2.1) circle (1.5pt);
\draw[color=black] (4.4,2.3) node {$z=a+bi$};
\draw[color=black] (1.64,1.25) node {$r$};
\draw[color=black] (3.7,1.0) node {$b$};
\draw[color=black] (1.8,-0.23) node {$a$};
\draw[color=black] (1.12,0.31) node {$\theta$};
\end{scriptsize}
\end{axis}
\end{tikzpicture}
\end{center}

As shown in the figure, any complex number $a+bi$ can also be expressed in polar coordinates $(r;\theta)$. 

\begin{enumerate}
\item Express in terms of $r$ and $\theta$:  $a=\answer{r\cos\theta}$, $b=\answer{r\sin\theta}$. 
\item Express in terms of $a$ and $b$: $r^2=\answer{a^2+b^2}$, $\tan\theta = \answer{\frac{b}{a}}$. 
\end{enumerate}

% $r(\cos\theta +i\sin\theta)$.  
%If $a+bi = r(\cos\theta +i\sin\theta)$, for real numbers $a$, $b$, $r$, and $\theta$, 

Practice converting among these forms, particularly for special angles.

\end{problem}

%$e^{c+di}=e^ce^{di}=e^c(\cos d +i\sin d)$
%$r(\cos \theta +i\sin \theta)= re^{i\theta}=e^{\ln r}e^{i\theta}=e^{\ln r+i\theta}$

% Checking 
%
%\begin{problem}
%Enter $1+2i+i^2$ here: $\answer{1+2i+i^2}$.
%
%Enter $1+2i+i^2$ here: $\answer{1+2\cdot i+i^2}$.
%\end{problem}


\begin{problem}
Convert the following complex numbers to polar coordinates with $r>0$ and $0\le\theta < 2\pi$:
\begin{center}
\begin{tabular}{c | c}
standard form & polar coordinates \\
\hline
  $i$ & $\left( \answer{1}; \answer{\pi/2} \right)$ \\
 $1 + i$  & $\left( \answer{\sqrt{2}}; \answer{\pi/4} \right)$ \\
 $-3$ & $\left( \answer{3}; \answer{\pi} \right)$ \\
 $2 - 2i$ & $\left( \answer{2\sqrt{2}}; \answer{7\pi/4} \right)$ \\
 $-\sqrt{3}-i$ & $\left( \answer{2}; \answer{7\pi/6} \right)$ \\
 $2 - 2i\sqrt{3}$ & $\left( \answer{4}; \answer{5\pi/3} \right)$ \\
 $-5i$ & $\left( \answer{5}; \answer{3\pi/2} \right)$ \\
% $2 - 3i$ & $\left( \answer{1}; \answer{\pi/2} \right)$ \\
% $-4 + 6i$ & $\left( \answer{1}; \answer{\pi/2} \right)$
 \end{tabular}
\end{center}
\end{problem}

\begin{problem}
Convert the following complex numbers to polar coordinates, this time with $r>0$ and $-\pi < \theta \le \pi$:
\begin{center}
\begin{tabular}{c | c}
standard form & polar coordinates \\
\hline
%  $i$ & $\left( \answer{1}; \answer{\pi/2} \right)$ \\
% $1 + i$  & $\left( \answer{\sqrt{2}}; \answer{\pi/4} \right)$ \\
 $-3$ & $\left( \answer{3}; \answer{\pi} \right)$ \\
 $2 - 2i$ & $\left( \answer{2\sqrt{2}}; \answer{-\pi/4} \right)$ \\
 $-\sqrt{3}-i$ & $\left( \answer{2}; \answer{-5\pi/6} \right)$ \\
 $2 - 2i\sqrt{3}$ & $\left( \answer{4}; \answer{-\pi/3} \right)$ \\
 $-5i$ & $\left( \answer{5}; \answer{-\pi/2} \right)$ \\
% $2 - 3i$ & $\left( \answer{1}; \answer{\pi/2} \right)$ \\
% $-4 + 6i$ & $\left( \answer{1}; \answer{\pi/2} \right)$
 \end{tabular}
\end{center}
\end{problem}




\begin{problem}
Convert the following complex numbers to standard form, $a+bi$: 

\begin{center}
\begin{tabular}{c | c}
polar coordinates & standard form \\
\hline
$(2; 30^\circ)$  & $\answer{2e^{i\pi/6}}$ \\
$(1; 135^\circ)$ & $\answer{1e^{3i\pi/4}}$ \\
$(3; 240^\circ)$ & $\answer{3e^{4i\pi/3}}$ \\
$(5; 120^\circ)$ & $\answer{5e^{2i\pi/3}}$ \\
 \end{tabular}
\end{center}
\end{problem}

%\begin{problem}
%Write $3 + 4i$, $1 + i$, and their product in polar form.  What do you notice?  
%\end{problem}
%
%\begin{problem}
%Write $(1 + i)^8$ in standard $a + bi$ form.  
%\end{problem}

\begin{problem}
%Choose a point in the complex plane, and find four different ways of using polar coordinates to describe it, using angles between $-2\pi$ and $2\pi$.  
Which of the following representations are unique?  
\begin{selectAll}
\choice[correct]{A complex number in standard form}
\choice[correct]{A complex number in rectangular coordinates}
\choice{A complex number in polar coordinates}
\choice{A fractional representation of a rational number}
\end{selectAll}
\begin{problem}
Correct!  And we saw some examples of non-uniqueness above.  

Here is another way of making statements like these: 

If $a+bi=c+di$, with $a$, $b$, $c$, and $d$ real numbers, then it \wordChoice{\choice[correct]{must}\choice{need not}} follow that $a=\answer{c}$ and $b=\answer{d}$. 

If $(R;\alpha) = (r;\beta)$, then it \wordChoice{\choice{must}\choice[correct]{need not}} follow that $R=\answer{r}$ and $\alpha=\answer{\beta}$. 

\begin{problem}
Correct!  

In contexts where both $R>0$ and $r>0$, $\alpha - \beta$ must be a multiple of $\answer{2\pi}$ radians.  

But some mathematical contexts allow polar coordinates with $r<0$.  And when $r=0$, the angle can be anything!
\end{problem}

\end{problem}

\end{problem}



\end{document}
