\documentclass[space,nooutcomes]{ximera} 

% For preamble materials

\usepackage{pgf,tikz}
\usepackage{mathrsfs}
\usetikzlibrary{arrows}
\usepackage{framed}
\usepackage{amsmath}
\pgfplotsset{compat=1.16}

\usepackage{numprint} % For printing large numbers with commas. 
\npthousandsep{,}


\graphicspath{
  {./}
  {quickQuestions/}
  {ximeraTutorial/}
}


\pdfOnly{\renewenvironment{image}[1][]{\begin{center}}{\end{center}}}

%%% This set of code is all of our user defined commands
\newcommand{\bysame}{\mbox{\rule{3em}{.4pt}}\,}
\newcommand{\N}{\mathbb N}
\newcommand{\C}{\mathbb C}
\newcommand{\W}{\mathbb W}
\newcommand{\Z}{\mathbb Z}
\newcommand{\Q}{\mathbb Q}
\newcommand{\R}{\mathbb R}
\newcommand{\A}{\mathbb A}
\newcommand{\D}{\mathcal D}
\newcommand{\F}{\mathcal F}
\newcommand{\ph}{\varphi}
\newcommand{\ep}{\varepsilon}
\newcommand{\aph}{\alpha}
\newcommand{\QM}{\begin{center}{\huge\textbf{?}}\end{center}}

\renewcommand{\le}{\leqslant}
\renewcommand{\ge}{\geqslant}
\renewcommand{\a}{\wedge}
\renewcommand{\v}{\vee}
\renewcommand{\l}{\ell}
\newcommand{\mat}{\mathsf}
\renewcommand{\vec}{\mathbf}
\renewcommand{\subset}{\subseteq}
\renewcommand{\supset}{\supseteq}
\renewcommand{\emptyset}{\varnothing}
\newcommand{\xto}{\xrightarrow}
\renewcommand{\qedsymbol}{$\blacksquare$}
\newcommand{\bibname}{References and Further Reading}
\renewcommand{\bar}{\protect\overline}
\renewcommand{\hat}{\protect\widehat}
\renewcommand{\tilde}{\widetilde}
\newcommand{\tri}{\triangle}
\newcommand{\minipad}{\vspace{1ex}}
\newcommand{\leftexp}[2]{{\vphantom{#2}}^{#1}{#2}}

%% More user defined commands
\renewcommand{\epsilon}{\varepsilon}
%\renewcommand{\theta}{\vartheta} %% only for kmath
\renewcommand{\l}{\ell}
\renewcommand{\d}{\, d}
\newcommand{\ddx}{\frac{d}{dx}}
\newcommand{\dydx}{\frac{dy}{dx}}


\usepackage{bigstrut}


\newenvironment{sectionOutcomes}{}{}

\usepackage{array}
%\setlength{\extrarowheight}{-.2cm}   % Commented out by Findell to fix table headings.  Was this for typesetting division?  
\newdimen\digitwidth
\settowidth\digitwidth{9}
\def~{\hspace{\digitwidth}}
\def\divrule#1#2{
\noalign{\moveright#1\digitwidth
\vbox{\hrule width#2\digitwidth}}}


\title{Polar Form}
\author{Brad Findell}
\begin{document}
\begin{abstract}
Exploring complex numbers in polar form.   
\end{abstract}
\maketitle



% The notation $\text{cis}\theta$ is a shorthand for $\cos\theta +i\sin\theta$, sometimes used in Precalculus texts. 

You have seen Euler's formula: $e^{i\theta}=\cos\theta +i\sin\theta$.  

Any complex number $a+bi$ can be expressed as $r(\cos\theta +i\sin\theta)$.  Thinking of the complex plane as ordered pairs of real numbers, the number $a+bi$ can be represented as the point $(a,b)$ or as a vector from the origin to $(a,b)$.  If $a+bi = r(\cos\theta +i\sin\theta)$, for real numbers $a$, $b$, $r$, and $\theta$, then we say that the complex number has rectangular coordinates $(a,b)$ and polar coordinates $(r;\theta)$.  Practice converting among these forms, particularly for special angles.


\begin{problem}


$a=r\cos\theta$, $b=r\sin\theta$. 

$a^2+b^2=r^2$, $\theta = \arctan\frac{b}{a}$. 

$e^{c+di}=e^ce^{di}=e^c(\cos d +i\sin d)$

$r(\cos \theta +i\sin \theta)= re^{i\theta}=e^{\ln r}e^{i\theta}=e^{\ln r+i\theta}$

\end{problem}

\begin{problem}
Enter $1+2i+i^2$ here: $\answer{1+2i+i^2}$.

Enter $1+2i+i^2$ here: $\answer{1+2\cdot i+i^2}$.
\end{problem}


\begin{problem}
Convert the following complex numbers to polar form with $r>0$ and $0\le\theta < 2\pi$:
\begin{center}
\begin{tabular}{c | c}
standard form & polar form \\
\hline
  $i$ & $\left( \answer{1}; \answer{\pi/2} \right)$ \\
 $1 + i$  & $\left( \answer{\sqrt{2}}; \answer{\pi/4} \right)$ \\
 $-3$ & $\left( \answer{3}; \answer{\pi} \right)$ \\
 $2 - 2i$ & $\left( \answer{2\sqrt{2}}; \answer{7\pi/4} \right)$ \\
 $-\sqrt{3}-i$ & $\left( \answer{2}; \answer{7\pi/6} \right)$ \\
 $2 - 2i\sqrt{3}$ & $\left( \answer{4}; \answer{5\pi/3} \right)$ \\
 $-5i$ & $\left( \answer{5}; \answer{3\pi/2} \right)$ \\
% $2 - 3i$ & $\left( \answer{1}; \answer{\pi/2} \right)$ \\
% $-4 + 6i$ & $\left( \answer{1}; \answer{\pi/2} \right)$
 \end{tabular}
\end{center}
\end{problem}

\begin{problem}
Convert the following complex numbers to polar form, this time with $r>0$ and $-\pi < \theta \le \pi$:
\begin{center}
\begin{tabular}{c | c}
standard form & polar form \\
\hline
%  $i$ & $\left( \answer{1}; \answer{\pi/2} \right)$ \\
% $1 + i$  & $\left( \answer{\sqrt{2}}; \answer{\pi/4} \right)$ \\
 $-3$ & $\left( \answer{3}; \answer{\pi} \right)$ \\
 $2 - 2i$ & $\left( \answer{2\sqrt{2}}; \answer{-\pi/4} \right)$ \\
 $-\sqrt{3}-i$ & $\left( \answer{2}; \answer{-5\pi/6} \right)$ \\
 $2 - 2i\sqrt{3}$ & $\left( \answer{4}; \answer{-\pi/3} \right)$ \\
 $-5i$ & $\left( \answer{5}; \answer{-\pi/2} \right)$ \\
% $2 - 3i$ & $\left( \answer{1}; \answer{\pi/2} \right)$ \\
% $-4 + 6i$ & $\left( \answer{1}; \answer{\pi/2} \right)$
 \end{tabular}
\end{center}
\end{problem}




\begin{problem}
Convert the following complex numbers to standard form, $a+bi$: 

\begin{center}
\begin{tabular}{c | c}
polar form & standard form \\
\hline
$(2; 30^\circ)$  & $\answer{2e^{i\pi/6}}$ \\
$(1; 135^\circ)$ & $\answer{1e^{3i\pi/4}}$ \\
$(3; 240^\circ)$ & $\answer{3e^{4i\pi/3}}$ \\
$(5; 120^\circ)$ & $\answer{5e^{2i\pi/3}}$ \\
 \end{tabular}
\end{center}
\end{problem}

\newpage 

\begin{problem}
Write $3 + 4i$, $1 + i$, and their product in polar form.  What do you notice?  
\vfill 
\end{problem}

\begin{problem}
Write $(1 + i)^8$ in standard $a + bi$ form.  
\vfill 
\end{problem}

\begin{problem}
Choose a point in the complex plane, and find four different ways of using polar coordinates to describe it, using angles between $-2\pi$ and $2\pi$.  
Do polar coordinates uniquely label a point in the plane?
Do rectangular coordinates uniquely label a point in the plane?
\vfill 
\end{problem}




\end{document}
