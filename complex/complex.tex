\documentclass[space,nooutcomes]{ximera} 

% For preamble materials

\usepackage{pgf,tikz}
\usepackage{mathrsfs}
\usetikzlibrary{arrows}
\usepackage{framed}
\usepackage{amsmath}
\pgfplotsset{compat=1.16}

\usepackage{numprint} % For printing large numbers with commas. 
\npthousandsep{,}


\graphicspath{
  {./}
  {quickQuestions/}
  {ximeraTutorial/}
}


\pdfOnly{\renewenvironment{image}[1][]{\begin{center}}{\end{center}}}

%%% This set of code is all of our user defined commands
\newcommand{\bysame}{\mbox{\rule{3em}{.4pt}}\,}
\newcommand{\N}{\mathbb N}
\newcommand{\C}{\mathbb C}
\newcommand{\W}{\mathbb W}
\newcommand{\Z}{\mathbb Z}
\newcommand{\Q}{\mathbb Q}
\newcommand{\R}{\mathbb R}
\newcommand{\A}{\mathbb A}
\newcommand{\D}{\mathcal D}
\newcommand{\F}{\mathcal F}
\newcommand{\ph}{\varphi}
\newcommand{\ep}{\varepsilon}
\newcommand{\aph}{\alpha}
\newcommand{\QM}{\begin{center}{\huge\textbf{?}}\end{center}}

\renewcommand{\le}{\leqslant}
\renewcommand{\ge}{\geqslant}
\renewcommand{\a}{\wedge}
\renewcommand{\v}{\vee}
\renewcommand{\l}{\ell}
\newcommand{\mat}{\mathsf}
\renewcommand{\vec}{\mathbf}
\renewcommand{\subset}{\subseteq}
\renewcommand{\supset}{\supseteq}
\renewcommand{\emptyset}{\varnothing}
\newcommand{\xto}{\xrightarrow}
\renewcommand{\qedsymbol}{$\blacksquare$}
\newcommand{\bibname}{References and Further Reading}
\renewcommand{\bar}{\protect\overline}
\renewcommand{\hat}{\protect\widehat}
\renewcommand{\tilde}{\widetilde}
\newcommand{\tri}{\triangle}
\newcommand{\minipad}{\vspace{1ex}}
\newcommand{\leftexp}[2]{{\vphantom{#2}}^{#1}{#2}}

%% More user defined commands
\renewcommand{\epsilon}{\varepsilon}
%\renewcommand{\theta}{\vartheta} %% only for kmath
\renewcommand{\l}{\ell}
\renewcommand{\d}{\, d}
\newcommand{\ddx}{\frac{d}{dx}}
\newcommand{\dydx}{\frac{dy}{dx}}


\usepackage{bigstrut}


\newenvironment{sectionOutcomes}{}{}

\usepackage{array}
%\setlength{\extrarowheight}{-.2cm}   % Commented out by Findell to fix table headings.  Was this for typesetting division?  
\newdimen\digitwidth
\settowidth\digitwidth{9}
\def~{\hspace{\digitwidth}}
\def\divrule#1#2{
\noalign{\moveright#1\digitwidth
\vbox{\hrule width#2\digitwidth}}}


\title{Complex Numbers}
\author{Brad Findell}
\begin{document}
\begin{abstract}
Problems exploring complex numbers.   
\end{abstract}
\maketitle


\begin{problem}
Find all complex solutions to each of the following equations: 
\[
x^3=1 \qquad x^4=1 \qquad x^6 = 1 \qquad x^8 = 1
\]
\vfill
\end{problem}
\newpage 

\begin{problem}
Let $\omega=-\tfrac{1}{2}+\tfrac{\sqrt{3}}{2}i$.  Compute $w^n$, for $n = -2, -1, 0, 1, 2, 3, 4$, and $134$.  
\vfill
\end{problem}

\begin{problem} 
Let $\alpha=\tfrac{\sqrt{2}}{2}+\tfrac{\sqrt{2}}{2}i$ Compute $\alpha^n$ for $n = -1, 0, 1, 2, \dots, 8$, and $143$.  
\vfill
\end{problem}

% Fifth roots of unity: $\beta = \frac{1}{4}\left(\sqrt{5}-1\right)+\frac{1}{4}i\sqrt{2}\sqrt{5+\sqrt{5}}$

\begin{problem}
Describe where and how you see modular arithmetic in the previous problems.
\vfill
\end{problem}

\newpage

% The notation $\text{cis}\theta$ is a shorthand for $\cos\theta +i\sin\theta$, sometimes used in Precalculus texts. 

You have seen Euler's formula: $e^{i\theta}=\cos\theta +i\sin\theta$.  Any complex number $a+bi$ can be expressed as $r(\cos\theta +i\sin\theta)$.  Thinking of the complex plane as ordered pairs of real numbers, the number $a+bi$ can be represented as the point $(a,b)$ or as a vector from the origin to
$(a,b)$.  If $a+bi = r(\cos\theta +i\sin\theta)$, for real numbers $a$, $b$, $r$, and $\theta$, then we say that the complex number has rectangular coordinates $(a,b)$ and polar coordinates $(r;\theta)$.  Practice converting among these forms, particularly for special angles.


\begin{problem}
Convert the following complex numbers to polar form: $i$, $1 + i$,  $-3$, $2 - 2i$, $-\sqrt{3}-i$, $2 - 3i$, $-4 + 6i$.  
\vfill 
\end{problem}

\begin{problem}
Convert the following complex numbers to rectangular form: $(2; 30)$, $(1; 135^\circ)$, 
$(3; 240^\circ)$, $(5; 27^\circ)$.  
\vfill 
\end{problem}

\newpage 

\begin{problem}
Write $3 + 4i$, $1 + i$, and their product in polar form.  What do you notice?  
\vfill 
\end{problem}

\begin{problem}
Write $(1 + i)^8$ in standard $a + bi$ form.  
\vfill 
\end{problem}

\begin{problem}
Choose a point in the complex plane, and find four different ways of using polar coordinates to describe it, using angles between $-2\pi$ and $2\pi$.  
Do polar coordinates uniquely label a point in the plane?
Do rectangular coordinates uniquely label a point in the plane?
\vfill 
\end{problem}

\newpage 


\begin{problem}
Assuming that $x$ and $y$ are real numbers, explain why the product $(x+iy)(x-iy)$ is always a nonnegative real number. 
\vfill
\end{problem}

Given a complex number $z=a+bi$, the absolute value of $z$ (also called the modulus of $z$) is given by $|z|=\sqrt{a^2+b^2}$.  The conjugate of $z$, denoted $\overline{z}$, is $a-bi$.  When a complex number is given in polar coordinates, the angle is sometimes called the \textbf{argument} of $z$, and it is common to choose $-\pi<\arg(z)\le \pi$.  


\begin{problem}
If you think of $z$ as a vector from the origin, what is a geometric interpretation of $|z|$?  How does it relate to $z\overline{z}$?  
\vfill 
\end{problem}

\begin{problem}
How does $|z+w|$ relate to $|z|$ and $|w|$?  Explain.  
\vfill 
\end{problem}

\begin{problem}
How does $|zw|$ relate to $|z|$ and $|w|$?  
\vfill 
\end{problem}

\begin{problem}
How do $\overline{zw}$, $\overline{z+w}$, and $\overline{z^{-1}}$ relate to $\overline{z}\,\overline{w}$, $\overline{z}+\overline{w}$, and $(\overline{z})^{-1}$?  What does this say about the algebra of complex conjugation?  
\vfill 
\end{problem}

\begin{problem}
Given a complex number $z=a+bi$, find (and simplify, if possible) $\overline{z}$, $z\overline{z}$, $z^{-1}$, $z+\overline{z}$, and $z-\overline{z}$.  
\vfill 
\end{problem}

\begin{problem}
Now suppose the complex number $z$ is on the unit circle.  Find $\overline{z}$, $z^{-1}$, $z\overline{z}$, $z+\overline{z}$, and $z-\overline{z}$.
\vfill 
\end{problem}

%\begin{problem}
%Use the angle addition identity to prove DeMoivre's theorem:  .
%\end{problem}

\newpage 


\begin{problem}
Describe (geometrically) the following transformations of the complex plane: 
\begin{enumerate}
\item Multiplication by $3$. 
\item Multiplication by $-1$. 
\item Addition of $2 + i$. 
\item Subtraction of $1 + 2i$.  
\item Multiplication by $i$. 
\item Complex conjugation. 
\item Multiplication by $1/2$.  
\end{enumerate}
\vfill 
\end{problem}

\begin{problem}
Describe how addition of $a + bi$ transforms the plane, and include the special cases that you described in a previous problem.
\vfill 
\end{problem}

\begin{problem}
Describe how multiplication by $r$ transforms the plane, when $r > 0$, and include the special cases that you described in a previous problem.  
\vfill 
\end{problem}

\begin{problem}
Prove, without using trigonometry, that multiplication by $i$ rotates the complex plane $90^{\circ}$ counter-clockwise.  Hint: Multiply $z=x+iy$ by $i$ and think about slope.
\vfill 
\end{problem}


\begin{problem}
Describe how multiplication by $\cos\theta + i\sin\theta$ transforms the plane, and include the special cases that you described in a previous problem.
\vfill 
\end{problem}


\newpage 


%\begin{problem}
%Use your calculator to compute $\cis 72 + \cis 144 + \cis 216 + \cis 288 + \cis 360$.
%\end{problem}

\begin{problem}
Use long division to simplify $\frac{x^{43}-1}{x-1}$.
\vfill 
\end{problem}

\begin{problem}
Simplify $a+ar+ar^2+\dots+ar^{n-1}$.  (Write it without dots.)
\vfill 
\end{problem}

\begin{problem}
Calculate $\sqrt{5+12i}$.  (Hint: Call it $a + bi$.)  
\vfill 
\end{problem}

\begin{problem}
Prove the zero product property for complex numbers.  Give an example of a number system in which the zero product property does not hold.
\vfill 
\end{problem}

\newpage 


\begin{problem}
Prove without calculation that $(a+bi)/(a-bi)$ always has absolute value 1.  
\vfill 
\end{problem}

%\begin{problem}
%Interpret the angle of the complex number $(z_1 - z_2 )/(z_1 - z_3)$.
%\end{problem}
%
%\begin{problem}
%Use absolute value to write an equation of a circle with center $z_0$ and radius $r$ in the complex plane. 
%\end{problem}
%
%\begin{problem}
%Given a complex number $z=a+bi$, the real part of $z$ is $a$, and the imaginary part is $b$.  We abbreviate these statements by writing $Re(z) = a$ and $Im(z) = b$.  Write formulas in terms of $z$ and $\overline{z}$ for $Re(z)$ and $Im(z)$.  
%\end{problem}
%
%\begin{problem}
%Use your formulas for $Re(z)$ and $Im(z)$ to write an equation in $z$ (without $x$ and $y$) for the line $y = 2x - 3$ in the complex plane.  Simplify the equation.  
%\end{problem}
%
%\begin{problem}
%Write an equation in $z$ for the line $ax+by =c$, where $a$, $b$, and $c$ are real numbers.  What happens when $a = 0$?  When $b = 0$? 
%\end{problem}


\begin{problem}
For positive real numbers $a$ and $b$, it is well known that $\sqrt{ab}=\sqrt{a}\sqrt{b}$.  Show that this rule does not always hold for complex numbers, and explain why.  
\vfill 
\end{problem}

\begin{problem}
Show that with the convention $-\pi<\arg{z}\le\pi$, the rule $\arg{zw}=\arg{z}+\arg{w}$ holds sometimes but not always.  Give examples of each.  Explain the advantages of thinking of $\arg{z}$ as an equivalence class of numbers that differ from one another by integer multiples of $2\pi$?.
\vfill 
\end{problem}
%
%\begin{problem}
%Show that if $p$ and $q$ are complex numbers and $k > 0$, then the graph of $\left|\frac{z-p}{z-q}\right|=k$ is a circle or a line in the complex plane.  Specify the conditions under which the graph is a line.  
%\end{problem}
%
%\begin{problem}
%Pick a complex number $w=a+bi$ where $a$ and $b$ are integers and $a\ne b$.  Let $z=w^2$.  Find the length of $z$.  What do you notice?  Explain how you can use this idea to generate some ``useful collections of numbers.''  Show that the method works and explain why it is necessary that $a$ and $b$ be integers with $a\ne b$.  
%\end{problem}
%



\end{document}
