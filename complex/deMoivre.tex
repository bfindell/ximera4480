\documentclass[space,nooutcomes]{ximera} 

% For preamble materials

\usepackage{pgf,tikz}
\usepackage{mathrsfs}
\usetikzlibrary{arrows}
\usepackage{framed}
\usepackage{amsmath}
\pgfplotsset{compat=1.16}

\usepackage{numprint} % For printing large numbers with commas. 
\npthousandsep{,}


\graphicspath{
  {./}
  {quickQuestions/}
  {ximeraTutorial/}
}


\pdfOnly{\renewenvironment{image}[1][]{\begin{center}}{\end{center}}}

%%% This set of code is all of our user defined commands
\newcommand{\bysame}{\mbox{\rule{3em}{.4pt}}\,}
\newcommand{\N}{\mathbb N}
\newcommand{\C}{\mathbb C}
\newcommand{\W}{\mathbb W}
\newcommand{\Z}{\mathbb Z}
\newcommand{\Q}{\mathbb Q}
\newcommand{\R}{\mathbb R}
\newcommand{\A}{\mathbb A}
\newcommand{\D}{\mathcal D}
\newcommand{\F}{\mathcal F}
\newcommand{\ph}{\varphi}
\newcommand{\ep}{\varepsilon}
\newcommand{\aph}{\alpha}
\newcommand{\QM}{\begin{center}{\huge\textbf{?}}\end{center}}

\renewcommand{\le}{\leqslant}
\renewcommand{\ge}{\geqslant}
\renewcommand{\a}{\wedge}
\renewcommand{\v}{\vee}
\renewcommand{\l}{\ell}
\newcommand{\mat}{\mathsf}
\renewcommand{\vec}{\mathbf}
\renewcommand{\subset}{\subseteq}
\renewcommand{\supset}{\supseteq}
\renewcommand{\emptyset}{\varnothing}
\newcommand{\xto}{\xrightarrow}
\renewcommand{\qedsymbol}{$\blacksquare$}
\newcommand{\bibname}{References and Further Reading}
\renewcommand{\bar}{\protect\overline}
\renewcommand{\hat}{\protect\widehat}
\renewcommand{\tilde}{\widetilde}
\newcommand{\tri}{\triangle}
\newcommand{\minipad}{\vspace{1ex}}
\newcommand{\leftexp}[2]{{\vphantom{#2}}^{#1}{#2}}

%% More user defined commands
\renewcommand{\epsilon}{\varepsilon}
%\renewcommand{\theta}{\vartheta} %% only for kmath
\renewcommand{\l}{\ell}
\renewcommand{\d}{\, d}
\newcommand{\ddx}{\frac{d}{dx}}
\newcommand{\dydx}{\frac{dy}{dx}}


\usepackage{bigstrut}


\newenvironment{sectionOutcomes}{}{}

\usepackage{array}
%\setlength{\extrarowheight}{-.2cm}   % Commented out by Findell to fix table headings.  Was this for typesetting division?  
\newdimen\digitwidth
\settowidth\digitwidth{9}
\def~{\hspace{\digitwidth}}
\def\divrule#1#2{
\noalign{\moveright#1\digitwidth
\vbox{\hrule width#2\digitwidth}}}


\title{DeMoivre's Theorem}
\author{Brad Findell}
\begin{document}
\begin{abstract}
Problems exploring DeMoivre's Theorem.   
\end{abstract}
\maketitle



%The notation cis theta is shorthand for  

You have seen Euler's formula: $e^{i\theta}=\cos\theta +i\sin\theta$, which can be derived from the Taylor series for $e^{x}$, $\cos x$, and $\sin x$ by separating the real and imaginary parts of $e^{i\theta}$.  Note that because 
$\cos^2\theta+\sin^2\theta=1$, the complex number $\cos\theta +i\sin\theta$ lies on the 
unit circle in the complex plane.  More generally, any complex number $a+bi$ can be expressed as $r(\cos\theta +i\sin\theta)$, with $r\ge 0$.  

Thinking of the complex plane as ordered pairs of real numbers, the number $a+bi$ can be represented as the point $(a,b)$ or as a vector from the origin to
$(a,b)$.  If $a+bi = r(\cos\theta +i\sin\theta)$, for real numbers $a$, $b$, $r$, and $\theta$, then we say that the complex number has rectangular coordinates $(a,b)$ and polar coordinates $(r;\theta)$. We call $r$ the magnitude and $\theta$ the angle (or argument) of the complex number. Practice converting among these forms, particularly for special angles.

%\begin{problem}
%Use the angle addition identity to prove DeMoivre's theorem:  .
%\end{problem}

\begin{problem}
From the algebra of exponents, it is clear that 
\[
e^{i\alpha}e^{i\beta}=e^{i(\alpha + \beta)} 
\]
With Euler's formula, this identity implies that 
\[
r_1(\cos\alpha + i\sin\alpha)r_2(\cos\beta + i\sin\beta)
=r_1 r_2\left(\cos(\alpha+\beta) + i\sin(\alpha+\beta)\right). 
\]
What does this identity tell us about the geometry of complex multiplication?  
\begin{freeResponse}
\begin{hint}
When multiplying complex numbers, simply multiply their magnitudes and add their angles (arguments).  
\end{hint}
\end{freeResponse}
\end{problem}

\begin{problem}
Multiply $a + bi$ times $c + di$ for the special case when $a=\cos\alpha$, $b=\sin\alpha$, 
$c=\cos\beta$, $d=\sin\beta$.  Explain how this idea could help you reconstruct certain trigonometry formulas, if you had forgotten them.  

Explain, alternatively, how you could use your knowledge of trigonometry to prove the angle-addition property of complex multiplication.  
\end{problem}

%
%\begin{problem}
%Prove the zero product property for complex numbers.  Give an example of a number system in which the zero product property does not hold.
%\vfill 
%\end{problem}
%
Let's step back for a moment to see whether we can generate the same understanding about complex multiplication without Euler's formula---and without trigonometry.  
\begin{problem}
Prove that multiplying by a complex number $w = a + bi$ is a composition of a rotation and a dilation and specify both the angle of rotation and the dilation factor.  Hints: For a generic complex number $z$, how do $za$ and $zbi$ relate to $z$?  How does $zw$ relate to $za$ and $zbi$?  
\end{problem}

\begin{problem}
Prove:  $(\cos\theta + i\sin\theta)^2=\cos 2\theta + i\sin 2\theta$.  Describe the formula in words.    
\end{problem}

\begin{problem}
Prove DeMoivre's Theorem:  $(\cos\theta + i\sin\theta)^n=\cos n\theta + i\sin n\theta$, for $n=1, 2, \dots$.  Hint: Use mathematical induction on $n$. 
\end{problem}

\subsection{Finding roots}
DeMoivre's theorem is often used to find roots of complex numbers.  
To derive the formula, observe first that, in polar form, complex numbers are not uniquely identified.  In particular, 
\[
\cos\theta + i\sin\theta =\cos (\theta+2k\pi) + i\sin (\theta+2k\pi), 
\]
for any integer $k$. Thus, by DeMoivre's theorem:  
\begin{align*}
\left[ \cos \left(\frac{\theta+2k\pi}{n}\right) 
+ i\sin \left(\frac{\theta+2k\pi}{n}\right) \right]^n 
&= \cos (\theta+2k\pi) + i\sin (\theta+2k\pi) \\
&= (\cos\theta + i\sin\theta), 
\end{align*}
which means that 
\[
\cos \left(\frac{\theta+2k\pi}{n}\right) 
+ i\sin \left(\frac{\theta+2k\pi}{n}\right)
\]
is an $n^\text{th}$ root of $(\cos\theta + i\sin\theta)$ for any integer $k$.  To ensure that these $n^\text{th}$ roots are distinct, we typically choose $k= 0, 1, \dots, n-1$.  This approach works for any complex number on the unit circle.  

Now we generalize this approach to find the $n^\text{th}$ roots any non-zero complex number $z=r(\cos\theta + i\sin\theta)$, where $r>0$.  Because $r$ is a positive real number, it has exactly one positive real $n^\text{th}$ root, called its ``principal $n^\text{th}$ root,'' which we can then unambiguously denote as $\sqrt[n]{r}$ or as $r^{1/n}$.  Then the $n$ distinct $n^\text{th}$ roots of $z$ are as follows: 
\[
\sqrt[n]{r}\left[\cos \left(\frac{\theta+2k\pi}{n}\right) 
+ i\sin \left(\frac{\theta+2k\pi}{n}\right)\right], \text{ for } k= 0, 1, \dots, n-1. 
\]
Notice that this means that the $n$ roots are neatly arranged on a circle of radius 
$\sqrt[n]{r}$, separated by a uniform angle of $\frac{2\pi}{n}$ radians.  

\begin{problem}
Use DeMoivre's Theorem to find: 
\begin{enumerate}
\item The $5^\text{th}$ roots 1.  
\item The $4^\text{th}$ roots $i$.  
\item The $3^\text{rd}$ roots $-8i$.  
\end{enumerate}
Note: You may express your answers in either radians or degrees.  
\end{problem}



\end{document}
