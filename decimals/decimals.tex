\documentclass[space,nooutcomes]{ximera} 

% For preamble materials

\usepackage{pgf,tikz}
\usepackage{mathrsfs}
\usetikzlibrary{arrows}
\usepackage{framed}
\usepackage{amsmath}
\pgfplotsset{compat=1.13}
%\pgfplotsset{compat=1.16}

\usepackage{numprint} % For printing large numbers with commas. 
\npthousandsep{,}


\graphicspath{
  {./}
  {quickQuestions/}
  {ximeraTutorial/}
}


\pdfOnly{\renewenvironment{image}[1][]{\begin{center}}{\end{center}}}

%%% This set of code is all of our user defined commands
\newcommand{\bysame}{\mbox{\rule{3em}{.4pt}}\,}
\newcommand{\N}{\mathbb N}
\newcommand{\C}{\mathbb C}
\newcommand{\W}{\mathbb W}
\newcommand{\Z}{\mathbb Z}
\newcommand{\Q}{\mathbb Q}
\newcommand{\R}{\mathbb R}
\newcommand{\A}{\mathbb A}
\newcommand{\D}{\mathcal D}
\newcommand{\F}{\mathcal F}
\newcommand{\ph}{\varphi}
\newcommand{\ep}{\varepsilon}
\newcommand{\aph}{\alpha}
\newcommand{\QM}{\begin{center}{\huge\textbf{?}}\end{center}}

\renewcommand{\le}{\leqslant}
\renewcommand{\ge}{\geqslant}
\renewcommand{\a}{\wedge}
\renewcommand{\v}{\vee}
\renewcommand{\l}{\ell}
\newcommand{\mat}{\mathsf}
\renewcommand{\vec}{\mathbf}
\renewcommand{\subset}{\subseteq}
\renewcommand{\supset}{\supseteq}
\renewcommand{\emptyset}{\varnothing}
\newcommand{\xto}{\xrightarrow}
\renewcommand{\qedsymbol}{$\blacksquare$}
\newcommand{\bibname}{References and Further Reading}
\renewcommand{\bar}{\protect\overline}
\renewcommand{\hat}{\protect\widehat}
\renewcommand{\tilde}{\widetilde}
\newcommand{\tri}{\triangle}
\newcommand{\minipad}{\vspace{1ex}}
\newcommand{\leftexp}[2]{{\vphantom{#2}}^{#1}{#2}}

%% More user defined commands
\renewcommand{\epsilon}{\varepsilon}
%\renewcommand{\theta}{\vartheta} %% only for kmath
\renewcommand{\l}{\ell}
\renewcommand{\d}{\, d}
\newcommand{\ddx}{\frac{d}{dx}}
\newcommand{\dydx}{\frac{dy}{dx}}


\usepackage{bigstrut}


\newenvironment{sectionOutcomes}{}{}

\usepackage{array}
%\setlength{\extrarowheight}{-.2cm}   % Commented out by Findell to fix table headings.  Was this for typesetting division?  
\newdimen\digitwidth
\settowidth\digitwidth{9}
\def~{\hspace{\digitwidth}}
\def\divrule#1#2{
\noalign{\moveright#1\digitwidth
\vbox{\hrule width#2\digitwidth}}}


\title{Rational Decimals}
\author{Brad Findell}
\begin{document}
\begin{abstract}
Exploring decimals and rational numbers.  
\end{abstract}
\maketitle



\subsection*{Investigations}
Investigate the decimal representations of fractions $a/b$, where $a$ and $b$ are integers and $b$ is not $0$.  Use paper and pencil for at least some of your exploration, so that you can better see what is going on.  Feel free to use calculators and spreadsheets as well.  Target most of your investigation on cases when $0 < a < b$.
\begin{enumerate}
\item You should know or be able to figure out (in your head) decimal equivalents of fractions with many small denominators (i.e., 2, 3, 4, 5, 6, 8, 9, 10, 11, 12, 16, and 20).  
\item Here is a nice relationship between twelfths and eighths:  $1/8\approx 0.12$ and $1/12\approx 0.08$.  Find other such pairs.
\item Explore the fractions $1/7, 2/7, \dots, 6/7$.  What do you notice about the repeating pattern?  
\item Also investigate thirteenths and seventeenths.  
\item Given a rational number as a fraction in lowest terms, how can you determine quickly whether its decimal representation will repeat or terminate?
\begin{enumerate}
\item If the decimal terminates, how many digits will it have?  
\item If the decimal repeats, what can you say about the number of digits in the repeating pattern?  
\end{enumerate}
\item For some fractions $1/b$ with repeating decimal representations, explore $10^n (\bmod\, b)$ and look for a connection with the long division that produces the repeating pattern.  
\end{enumerate}

\subsection*{Proofs}
\begin{enumerate}
\item Explain why the decimal representation of a rational number must either terminate or repeat.  
\item Explain why every terminating decimal is a rational number.  
\item Explain why every repeating decimal is a rational number.  
\item Make and prove general statements about sums and products of rational and irrational numbers. (Be careful about 0.) 
\end{enumerate}

\subsection*{Related Common Core State Standards}
\begin{itemize}
\item 7.NS.2.d.  Convert a rational number to a decimal using long division; know that the decimal form of a rational number terminates in 0s or eventually repeats.

\item 8.NS.1. Know that numbers that are not rational are called irrational.  Understand informally that every number has a decimal expansion; for rational numbers show that the decimal expansion repeats eventually, and convert a decimal expansion which repeats eventually into a rational number.

\item N.RN.3. Explain why the sum or product of two rational numbers is rational; that the sum of a rational number and an irrational number is irrational; and that the product of a nonzero rational number and an irrational number is irrational.
\end{itemize}

\end{document}
