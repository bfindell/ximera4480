\documentclass{ximera}
%\documentclass[space,handout,nooutcomes]{ximera}

\title{Quick Questions}

\begin{document}
\begin{abstract}
Many teachers have quick answers to the following questions.  
\end{abstract}
\maketitle

Please provide quick answers and one-sentence explanations, when requested.  Answer off the top of your head, \textbf{without a calculator}, and spend \textbf{no more than 40 minutes} on these. 

%
%\begin{description}
%
%\end{description}

\begin{question}
Evaluate $-x^2$ when $x=9$.
\begin{freeResponse}
\end{freeResponse}
\end{question}

\begin{question}
Evaluate $x^{-2}$ when $x=9$.
\begin{freeResponse}
\end{freeResponse}
\end{question}

\begin{question}
Evaluate $x^{1/2}$ when $x=9$.
\begin{freeResponse}
\end{freeResponse}
\end{question}

\begin{question}
Evaluate $\frac{2}{0}$ and explain your answer. 
\begin{freeResponse}
\end{freeResponse}
\end{question}

\begin{question}
Evaluate $\frac{0}{0}$ and explain your answer. 
\begin{freeResponse}
\end{freeResponse}
\end{question}

\begin{question}
Evaluate $\frac{0}{2}$ and explain your answer. 
\begin{freeResponse}
\end{freeResponse}
\end{question}

\begin{question}
Is $0$ even, odd, neither, or both?  Explain.  
\begin{freeResponse}
\end{freeResponse}
\end{question}

\begin{question}
Give another explanation for the previous question.   
\begin{freeResponse}
\end{freeResponse}
\end{question}

\begin{question}
Is $\sqrt{4}=\pm2$?  Explain.  
\begin{freeResponse}
\end{freeResponse}
\end{question}

\begin{question}
To divide fractions, is it okay to convert to a common denominator and then ignore the denominators and divide the numerators?  Explain. 
\begin{freeResponse}
\end{freeResponse}
\end{question}

\begin{question}
To divide fractions, it is okay to divide the numerators and divide the denominators?  Explain.  
\begin{freeResponse}
\end{freeResponse}
\end{question}

\begin{question}
Write a ``story problem'' for $1\frac{3}{4}\div\frac{1}{2}$.  
\begin{freeResponse}
\end{freeResponse}
\end{question}

\begin{question}
Is $15 \equiv 7 \pmod 4$?  Explain.  
\begin{freeResponse}
\end{freeResponse}
\end{question}

\begin{question}
Is $2 \equiv 17 \pmod 5$?  Explain.  
\begin{freeResponse}
\end{freeResponse}
\end{question}

For the following three questions, suppose $f$ is a function with a domain and range that are both subsets of the real numbers and that $f(3)=2$.  Based on this information:  
\begin{question}
Where is the $3$?  \begin{freeResponse}\end{freeResponse}
\end{question}
\begin{question}
Where is the $2$? \begin{freeResponse}\end{freeResponse}
\end{question}
\begin{question}
Where is the $f(3)$?  \begin{freeResponse}\end{freeResponse}
\end{question}

\begin{question}
Is $0.99999\dots = 1$?  Explain.    
\begin{freeResponse}
\end{freeResponse}
\end{question}

\begin{question}
Why is $a^0 = 1$.  Does it matter what $a$ is?  
\begin{freeResponse}
\end{freeResponse}
\end{question}

\begin{question}
Why is $a^{-n} = \frac{1}{a^n}$.  Does it matter what $a$ is?  Does it matter what $n$ is?  
\begin{freeResponse}
\end{freeResponse}
\end{question}

\begin{question}
What does it mean for a number to be irrational?    
\begin{freeResponse}
\end{freeResponse}
\end{question}

\begin{question}
How long did you spend on these questions?  
\begin{freeResponse}
\end{freeResponse}
\end{question}

\end{document}

