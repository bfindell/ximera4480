\documentclass[10.5pt]{report}
\usepackage{fullpage}
\usepackage{multirow}
\usepackage{enumerate}
\usepackage{graphicx}
\usepackage{wrapfig}
\usepackage{amsfonts}
%\usepackage{minipage}

\global\topskip=0pt

\pagestyle{empty}

\renewcommand{\le}{\leqslant}
\renewcommand{\ge}{\geqslant}

\newtheorem{theorem}{Theorem}
\newtheorem{prob}{Problem}
\newtheorem{exer}{Exercise}
\newtheorem{lemma}[theorem]{Lemma}
\newtheorem{proposition}[theorem]{Proposition}
\newtheorem{corollary}[theorem]{Corollary}

\newenvironment{proof}[1][Proof]{\begin{trivlist}
\item[\hskip \labelsep {\bfseries #1}]}{\end{trivlist}}
\newenvironment{definition}[1][Definition]{\begin{trivlist}
\item[\hskip \labelsep {\bfseries #1}]}{\end{trivlist}}
\newenvironment{example}[1][Example]{\begin{trivlist}
\item[\hskip \labelsep {\bfseries #1}]}{\end{trivlist}}
\newenvironment{remark}[1][Remark]{\begin{trivlist}
\item[\hskip \labelsep {\bfseries #1}]}{\end{trivlist}}

\newcommand{\qed}{\nobreak \ifvmode \relax \else
      \ifdim\lastskip<1.5em \hskip-\lastskip
      \hskip1.5em plus0em minus0.5em \fi \nobreak
      \vrule height0.75em width0.5em depth0.25em\fi}




\begin{document}
\pagenumbering{gobble}



\section*{Activity 1b.  More Modular Arithmetic \\ \small{Math 4480, Autumn 2019}}

Modular arithmetic is usually constructed as equivalence classes of integers, and described informally as ``clock arithmetic'' or as ``arithmetic of remainders.''  In both cases, there are no points ``between'' consecutive numbers, such as 0 and 1.  

\subsection*{Background}
\paragraph{Division Theorem} Given integers $a$ and $b$ with $b\neq 0$, there exist unique integers $q$ (the quotient) and $r$ (the remainder) such that $a = bq + r$, with $0 \leq r < b$.
 
\paragraph{Definition of Modular Arithmetic}  Let $n$, $a$, and $b$ be integers, with $n > 0$.  Then $a$ and $b$ are congruent modulo $n$, written $a\equiv b\pmod{n}$, if $n$ divides the difference $a-b$, that is, if there exists an integer $k$ such that $a-b = kn$.
  
Examples:  
$$15\equiv 3 \pmod{4}$$

$$7\equiv  22 \pmod{5}$$

\paragraph{Theorem 1}
For integers $a$ and $b$, $a\equiv b \pmod{n}$ if and only if $a$ and $b$ have the same nonnegative remainder after dividing by $n$.

\paragraph{Theorem 2}
Let $a, b, c, d, k$ and $n$ be integers with $k>1$ and $n > 1$.
\begin{enumerate}[(a)]
\item $a\equiv a \pmod{n}$
\item If $a\equiv b \pmod{n}$, then $b\equiv a \pmod{n}$.  
\item If $a\equiv b \pmod{n}$ and $b\equiv c \pmod{n}$, then $a\equiv c \pmod{n}$.
\item If $a\equiv b \pmod{n}$ and $c\equiv d \pmod{n}$, then $a+c\equiv b+d \pmod{n}$ and $ac\equiv bd \pmod{n}$
\item If $a\equiv b \pmod{n}$, then $a^k\equiv b^k \pmod{n}$
\end{enumerate}

\subsection*{Investigations}
\begin{enumerate}
\item Most adults know divisibility rules for 2, 5, and 10.  Some adults know divisibility rules for 3, 4, 9, and even 11.  For $n = 2^k, 3, 5^k, 9$, and $11$, where $k$ is a counting number, recast the divisibility rules as ways to find remainders modulo $n$.
\item Which numbers have multiplicative inverses mod $n$?  What does it mean to be a multiplicative inverse?  
\end{enumerate}

\subsection*{Problems}
\begin{enumerate}
\item Prove Theorem 1 above.
\item Prove Theorem 2 above. 
\item Explain how Theorem 2 establishes that ``equivalence mod $n$'' is indeed an equivalence relation. 
\item Explain how Theorem 2 establishes that the operations of addition, multiplication, and exponentiation are ``well defined.''
\item Prove that your divisibility rules above work for finding remainders modulo $n$.  
\end{enumerate}

\end{document}
