\documentclass[space,nooutcomes]{ximera} 

% For preamble materials

\usepackage{pgf,tikz}
\usepackage{mathrsfs}
\usetikzlibrary{arrows}
\usepackage{framed}
\usepackage{amsmath}
\pgfplotsset{compat=1.13}
%\pgfplotsset{compat=1.16}

\usepackage{numprint} % For printing large numbers with commas. 
\npthousandsep{,}


\graphicspath{
  {./}
  {quickQuestions/}
  {ximeraTutorial/}
}


\pdfOnly{\renewenvironment{image}[1][]{\begin{center}}{\end{center}}}

%%% This set of code is all of our user defined commands
\newcommand{\bysame}{\mbox{\rule{3em}{.4pt}}\,}
\newcommand{\N}{\mathbb N}
\newcommand{\C}{\mathbb C}
\newcommand{\W}{\mathbb W}
\newcommand{\Z}{\mathbb Z}
\newcommand{\Q}{\mathbb Q}
\newcommand{\R}{\mathbb R}
\newcommand{\A}{\mathbb A}
\newcommand{\D}{\mathcal D}
\newcommand{\F}{\mathcal F}
\newcommand{\ph}{\varphi}
\newcommand{\ep}{\varepsilon}
\newcommand{\aph}{\alpha}
\newcommand{\QM}{\begin{center}{\huge\textbf{?}}\end{center}}

\renewcommand{\le}{\leqslant}
\renewcommand{\ge}{\geqslant}
\renewcommand{\a}{\wedge}
\renewcommand{\v}{\vee}
\renewcommand{\l}{\ell}
\newcommand{\mat}{\mathsf}
\renewcommand{\vec}{\mathbf}
\renewcommand{\subset}{\subseteq}
\renewcommand{\supset}{\supseteq}
\renewcommand{\emptyset}{\varnothing}
\newcommand{\xto}{\xrightarrow}
\renewcommand{\qedsymbol}{$\blacksquare$}
\newcommand{\bibname}{References and Further Reading}
\renewcommand{\bar}{\protect\overline}
\renewcommand{\hat}{\protect\widehat}
\renewcommand{\tilde}{\widetilde}
\newcommand{\tri}{\triangle}
\newcommand{\minipad}{\vspace{1ex}}
\newcommand{\leftexp}[2]{{\vphantom{#2}}^{#1}{#2}}

%% More user defined commands
\renewcommand{\epsilon}{\varepsilon}
%\renewcommand{\theta}{\vartheta} %% only for kmath
\renewcommand{\l}{\ell}
\renewcommand{\d}{\, d}
\newcommand{\ddx}{\frac{d}{dx}}
\newcommand{\dydx}{\frac{dy}{dx}}


\usepackage{bigstrut}


\newenvironment{sectionOutcomes}{}{}

\usepackage{array}
%\setlength{\extrarowheight}{-.2cm}   % Commented out by Findell to fix table headings.  Was this for typesetting division?  
\newdimen\digitwidth
\settowidth\digitwidth{9}
\def~{\hspace{\digitwidth}}
\def\divrule#1#2{
\noalign{\moveright#1\digitwidth
\vbox{\hrule width#2\digitwidth}}}


\title{Modular Arithmetic}
\author{Brad Findell}
\begin{document}
\begin{abstract}
Experience building activities in modular arithmetic.  
\end{abstract}
\maketitle

What is abstract algebra about?  Algebraic structures.  By the end of the term, you should have a sense of what that means.  Why is abstract algebra an important course for secondary teachers?  By the end of the term, you should have developed meaningful and precise answers to questions such as the following: 

\begin{itemize}
\item What is an inverse? 
\item What is an identity?
\item What does 1/2 mean?
\item What is a square root?
\item Why do exponents work as they do?
\item What does it mean to solve an equation?
\end{itemize}

\subsection*{Hints for Success}
\begin{itemize}
\item Pay attention to the operations.  Many of the operations you deal with will be called addition or multiplication, but not all additions are created equal, and the same goes for multiplication.  (What does this mean?)
\item Build up a collection of examples that you are familiar and comfortable with, regarding the objects, operations, and properties.  
\item Do lots of calculations in order to develop this comfort level and so that you might begin to notice patterns and regularities. 
\item Use operation tables and other methods of organizing your calculations.
\end{itemize}


\subsection*{Exploration}

\begin{problem}
Find a quick way to show that each of the following statements is false: 
\begin{enumerate}
\item $378 + 295 = 674$
\item $33 \times 47 = 1,552$
\item $65,537$ is a perfect square.
\item The 1922 presidential election was won by Calvin Coolidge.
\item $12,962,734$ is a perfect square.
\item $243,981,541 \times 3,617,192 = 882,528,068,252,872$
\end{enumerate}
\begin{freeResponse}
\end{freeResponse}
\end{problem}

In this class, we will use $Z_n$ to denote the integers modulo $n$.  The elements will usually be denoted $0,\ 1,\ 2,\ 3,\ \dots,\ n - 1$, and arithmetic will be performed modulo $n$.  

\begin{problem}
Food for thought:  How many of the following can you ``find'' in $Z_7$?  
\begin{hint}
What might $1/2$ mean in $Z_7$?  What would it mean for a number to be a square root of $2$?  
\end{hint}
\begin{hint}
The symbol $1/2$ means a solution to the equation $2x = 1$.  How might you solve this equation?  A square root of $2$ is a number whose square is $2$.  The notation $2^{-1}$ means the multiplicative inverse of $2$.  In $Z_n$, the na\"ive approach is to try all the possibilities.
\end{hint}
\[
4\cdot 5,\ 2 - 6,\ 1/2,\ 2/5,\ \sqrt{2},\ \sqrt{-3},\ \sqrt{-1},\ \sqrt[3]{6},\ 2^{-4}
\]
\begin{freeResponse}
\end{freeResponse}
\end{problem}

\begin{problem}
Answer each of the following questions: 
\begin{enumerate}
\item 90 days from Wednesday, Sept. 1, 2006 is what day of the week?
\item On what day of the week will Sept. 1 be next year?
\item Six weeks from Sept. 1 will be what date?
\item If the sum of the digits of a 5-digit number $n$ is 29, can $n$ be a perfect square?
\item Does $x^2 - 5y = 27$ have integer solutions?
\item Suppose $i^2 = -1$.  Compute $i^{7347}$.
\item It is now 9:00 o'clock.  In 7 hours it will be what time?
\item A machine can process a batch of widgets every five hours.  If the machine begins processing the first batch at midnight, what time will it be after the 15th batch is completed?
\item Each hour an object rotates $135^\circ$.  In how many hours will it return to its starting position?  
\end{enumerate}
\begin{freeResponse}
\end{freeResponse}
\end{problem}

\begin{problem}
Watch for opportunities to use modular arithmetic in your mathematical work or your day-to-day life.  Describe one such use.
\begin{freeResponse}
\end{freeResponse}
\end{problem}

\begin{problem}
More food for thought:  How many of the following can you ``find'' in $Z_{11}$? In $Z_{15}$? In $Z_9$?  Any conjectures? 
\[
7 + 8,\ 4 - 7,\ 3\cdot 5,\ 5^2,\ 5^3,\ 1/5,\ 3/8,\ \sqrt{3},\ \sqrt{-2},\ \sqrt{-6},\ 5^{-4} 
\]
\begin{freeResponse}
\end{freeResponse}
\end{problem}

\begin{problem}
Which elements in $Z_n$ have multiplicative inverses in $Z_n$?  
Hints:  Make multiplication tables for $Z_n$ for a variety of values of $n$.   Use your multiplication table to find multiplicative inverses.  Make conjectures and then check your conjecture for other elements and in other multiplication tables? 
\begin{freeResponse}
\end{freeResponse}
\end{problem}



\end{document}

