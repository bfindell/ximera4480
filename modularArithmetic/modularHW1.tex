\documentclass[space,nooutcomes,handout]{ximera} 

% For preamble materials

\usepackage{pgf,tikz}
\usepackage{mathrsfs}
\usetikzlibrary{arrows}
\usepackage{framed}
\usepackage{amsmath}
\pgfplotsset{compat=1.13}
%\pgfplotsset{compat=1.16}

\usepackage{numprint} % For printing large numbers with commas. 
\npthousandsep{,}


\graphicspath{
  {./}
  {quickQuestions/}
  {ximeraTutorial/}
}


\pdfOnly{\renewenvironment{image}[1][]{\begin{center}}{\end{center}}}

%%% This set of code is all of our user defined commands
\newcommand{\bysame}{\mbox{\rule{3em}{.4pt}}\,}
\newcommand{\N}{\mathbb N}
\newcommand{\C}{\mathbb C}
\newcommand{\W}{\mathbb W}
\newcommand{\Z}{\mathbb Z}
\newcommand{\Q}{\mathbb Q}
\newcommand{\R}{\mathbb R}
\newcommand{\A}{\mathbb A}
\newcommand{\D}{\mathcal D}
\newcommand{\F}{\mathcal F}
\newcommand{\ph}{\varphi}
\newcommand{\ep}{\varepsilon}
\newcommand{\aph}{\alpha}
\newcommand{\QM}{\begin{center}{\huge\textbf{?}}\end{center}}

\renewcommand{\le}{\leqslant}
\renewcommand{\ge}{\geqslant}
\renewcommand{\a}{\wedge}
\renewcommand{\v}{\vee}
\renewcommand{\l}{\ell}
\newcommand{\mat}{\mathsf}
\renewcommand{\vec}{\mathbf}
\renewcommand{\subset}{\subseteq}
\renewcommand{\supset}{\supseteq}
\renewcommand{\emptyset}{\varnothing}
\newcommand{\xto}{\xrightarrow}
\renewcommand{\qedsymbol}{$\blacksquare$}
\newcommand{\bibname}{References and Further Reading}
\renewcommand{\bar}{\protect\overline}
\renewcommand{\hat}{\protect\widehat}
\renewcommand{\tilde}{\widetilde}
\newcommand{\tri}{\triangle}
\newcommand{\minipad}{\vspace{1ex}}
\newcommand{\leftexp}[2]{{\vphantom{#2}}^{#1}{#2}}

%% More user defined commands
\renewcommand{\epsilon}{\varepsilon}
%\renewcommand{\theta}{\vartheta} %% only for kmath
\renewcommand{\l}{\ell}
\renewcommand{\d}{\, d}
\newcommand{\ddx}{\frac{d}{dx}}
\newcommand{\dydx}{\frac{dy}{dx}}


\usepackage{bigstrut}


\newenvironment{sectionOutcomes}{}{}

\usepackage{array}
%\setlength{\extrarowheight}{-.2cm}   % Commented out by Findell to fix table headings.  Was this for typesetting division?  
\newdimen\digitwidth
\settowidth\digitwidth{9}
\def~{\hspace{\digitwidth}}
\def\divrule#1#2{
\noalign{\moveright#1\digitwidth
\vbox{\hrule width#2\digitwidth}}}


\title{Modular Arithmetic, Homework 1}
\author{Brad Findell}
\begin{document}
\begin{abstract}
Experience building activities in modular arithmetic.  
\end{abstract}
\maketitle

\begin{problem}
Answer each of the following questions, and indicate the modulus that you used.  
\begin{enumerate}
\item 90 days from Tuesday, August 22, 2023 is what day of the week?  $\answer[format=string]{monday}$, modulo $\answer{7}$
\item On what day of the week will August 22, 2024 be? $\answer[format=string]{wednesday}$, modulo $\answer{7}$
\item Four weeks from August 22, 2023 will be what day of the month? September $\answer{13}$, modulo $\answer{31}$
%\item If the sum of the digits of a 5-digit number $n$ is 29, can $n$ be a perfect square?
%\item Does $x^2 - 5y = 27$ have integer solutions?
\item Suppose $i^2 = -1$.  Compute $i^{7347}$. $\answer{-i}$, modulo $\answer{4}$
\item It is now 9:00 o'clock.  In 7 hours it will be what time? $\answer{4}$, modulo $\answer{12}$
%\item A machine can process a batch of widgets every five hours.  If the machine begins processing the first batch at midnight, what time will it be after the 15th batch is completed?
%\item Each hour an object rotates $135^\circ$.  In how many hours will it return to its starting position?  
\end{enumerate}
\end{problem}

\subsection*{More food for thought.}  For the following calculations in $\mathbb{Z}_{n} = \{0, 1, \dots, n-1\}$, find sensible values.  If there is not a unique answer, indicate the number of answers by typing the word ``none'', ``two'', or ``three'' (without quotes). 
\begin{problem}
In $\mathbb{Z}_{11}$ find the following. 
\begin{enumerate}
\item $7 + 8 =\answer{4}$
\item $4 - 7 =\answer{8}$
\item $3\cdot 5 =\answer{4}$
\item $5^2 =\answer{3}$
\item $5^3 =\answer{4}$ 
\item $1/5 =\answer{9}$
\item $3/8 =\answer{10}$ 
\item $\sqrt{3} =\answer[format=string]{two}$ % 5 or 6
\item $\sqrt{-2} =\answer[format=string]{two}$  % 3 or 8
\item $\sqrt{-6} =\answer[format=string]{two}$  % 4 or 7
\item  $5^{-4}  =\answer{5}$ 
\end{enumerate}
\begin{problem}
In $\mathbb{Z}_{11}$, write your answers in increasing order: 
\begin{enumerate}
\item $\sqrt{3} =\answer{5}$ or $\answer{6}$ % 5 or 6
\item $\sqrt{-2} =\answer{3}$ or $\answer{8}$  % 3 or 8
\item $\sqrt{-6} =\answer{4}$ or $\answer{7}$  % 4 or 7
\end{enumerate}
\end{problem}
\end{problem}

\begin{problem}
In $\mathbb{Z}_{15}$ find the following. 
\begin{enumerate}
\item $7 + 8 =\answer{0}$
\item $4 - 7 =\answer{12}$
\item $3\cdot 5 =\answer{0}$
\item $5^2 =\answer{10}$
\item $5^3 =\answer{5}$ 
\item $1/5 =\answer[format=string]{none}$
\item $3/8 =\answer{10}$ 
\item $\sqrt{3} =\answer[format=string]{none}$
\item $\sqrt{-2} =\answer[format=string]{none}$  
\item $\sqrt{-6} =\answer[format=string]{none}$  
\item  $5^{-4}  =\answer[format=string]{none}$ 
\end{enumerate}
\end{problem}

\begin{problem}
In $\mathbb{Z}_{9}$ find the following. 
\begin{enumerate}
\item $7 + 8 =\answer{6}$
\item $4 - 7 =\answer{6}$
\item $3\cdot 5 =\answer{6}$
\item $5^2 =\answer{7}$
\item $5^3 =\answer{8}$ 
\item $1/5 =\answer{2}$
\item $3/8 =\answer{6}$ 
\item $\sqrt{3} =\answer[format=string]{none}$ 
\item $\sqrt{-2} =\answer[format=string]{two}$  % 4 or 5
\item $\sqrt{-6} =\answer[format=string]{none}$ 
\item  $5^{-4}  =\answer{7}$ 
\end{enumerate}
\begin{problem}
In $\mathbb{Z}_{9}$, write your answers in increasing order: 
\begin{enumerate}
\item $\sqrt{-2} =\answer{4}$ or $\answer{5}$  % 4 or 5
\end{enumerate}
\end{problem}
\end{problem}


\end{document}

