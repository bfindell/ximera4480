\documentclass[space,nooutcomes]{ximera} 

% For preamble materials

\usepackage{pgf,tikz}
\usepackage{mathrsfs}
\usetikzlibrary{arrows}
\usepackage{framed}
\usepackage{amsmath}
\pgfplotsset{compat=1.16}

\usepackage{numprint} % For printing large numbers with commas. 
\npthousandsep{,}


\graphicspath{
  {./}
  {quickQuestions/}
  {ximeraTutorial/}
}


\pdfOnly{\renewenvironment{image}[1][]{\begin{center}}{\end{center}}}

%%% This set of code is all of our user defined commands
\newcommand{\bysame}{\mbox{\rule{3em}{.4pt}}\,}
\newcommand{\N}{\mathbb N}
\newcommand{\C}{\mathbb C}
\newcommand{\W}{\mathbb W}
\newcommand{\Z}{\mathbb Z}
\newcommand{\Q}{\mathbb Q}
\newcommand{\R}{\mathbb R}
\newcommand{\A}{\mathbb A}
\newcommand{\D}{\mathcal D}
\newcommand{\F}{\mathcal F}
\newcommand{\ph}{\varphi}
\newcommand{\ep}{\varepsilon}
\newcommand{\aph}{\alpha}
\newcommand{\QM}{\begin{center}{\huge\textbf{?}}\end{center}}

\renewcommand{\le}{\leqslant}
\renewcommand{\ge}{\geqslant}
\renewcommand{\a}{\wedge}
\renewcommand{\v}{\vee}
\renewcommand{\l}{\ell}
\newcommand{\mat}{\mathsf}
\renewcommand{\vec}{\mathbf}
\renewcommand{\subset}{\subseteq}
\renewcommand{\supset}{\supseteq}
\renewcommand{\emptyset}{\varnothing}
\newcommand{\xto}{\xrightarrow}
\renewcommand{\qedsymbol}{$\blacksquare$}
\newcommand{\bibname}{References and Further Reading}
\renewcommand{\bar}{\protect\overline}
\renewcommand{\hat}{\protect\widehat}
\renewcommand{\tilde}{\widetilde}
\newcommand{\tri}{\triangle}
\newcommand{\minipad}{\vspace{1ex}}
\newcommand{\leftexp}[2]{{\vphantom{#2}}^{#1}{#2}}

%% More user defined commands
\renewcommand{\epsilon}{\varepsilon}
%\renewcommand{\theta}{\vartheta} %% only for kmath
\renewcommand{\l}{\ell}
\renewcommand{\d}{\, d}
\newcommand{\ddx}{\frac{d}{dx}}
\newcommand{\dydx}{\frac{dy}{dx}}


\usepackage{bigstrut}


\newenvironment{sectionOutcomes}{}{}

\usepackage{array}
%\setlength{\extrarowheight}{-.2cm}   % Commented out by Findell to fix table headings.  Was this for typesetting division?  
\newdimen\digitwidth
\settowidth\digitwidth{9}
\def~{\hspace{\digitwidth}}
\def\divrule#1#2{
\noalign{\moveright#1\digitwidth
\vbox{\hrule width#2\digitwidth}}}


\title{More Modular Arithmetic}
\author{Brad Findell}
\begin{document}
\begin{abstract}
More activities in modular arithmetic.  
\end{abstract}
\maketitle


Modular arithmetic is usually described informally as ``clock arithmetic'' or as ``arithmetic of remainders.''  The system is constructed formally as equivalence classes of integers,   In both cases, there are no points ``between'' consecutive numbers, such as 0 and 1.  

\subsection*{Background}
\paragraph{Division Theorem} Given integers $a$ and $b$ with $b\neq 0$, there exist unique integers $q$ (the quotient) and $r$ (the remainder) such that $a = bq + r$, with $0 \leq r < b$.
 
\paragraph{Definition of Modular Arithmetic}  Let $n$, $a$, and $b$ be integers, with $n > 0$.  Then $a$ and $b$ are congruent modulo $n$, written $a\equiv b\pmod{n}$, if $n$ divides the difference $a-b$, that is, if there exists an integer $k$ such that $a-b = kn$.
  
Examples:  
$$15\equiv 3 \pmod{4}$$

$$7\equiv  22 \pmod{5}$$

\paragraph{Theorem 1}
For integers $a$ and $b$, $a\equiv b \pmod{n}$ if and only if $a$ and $b$ have the same nonnegative remainder after dividing by $n$.

\paragraph{Theorem 2}
Let $a, b, c, d, k$ and $n$ be integers with $k>1$ and $n > 1$.
\begin{enumerate}%[(a)]
\item $a\equiv a \pmod{n}$
\item If $a\equiv b \pmod{n}$, then $b\equiv a \pmod{n}$.  
\item If $a\equiv b \pmod{n}$ and $b\equiv c \pmod{n}$, then $a\equiv c \pmod{n}$.
\item If $a\equiv b \pmod{n}$ and $c\equiv d \pmod{n}$, then $a+c\equiv b+d \pmod{n}$ and $ac\equiv bd \pmod{n}$
\item If $a\equiv b \pmod{n}$, then $a^k\equiv b^k \pmod{n}$
\end{enumerate}

\subsection*{Problems}

\begin{problem}
Most adults know divisibility rules for 2, 5, and 10.  Some adults know divisibility rules for 3, 4, 9, and even 11.  For $n = 2^k, 3, 5^k, 6, 9$, and $11$, where $k$ is a counting number, generalize the divisibility rules so that they are ``remainder rules'' for finding remainders modulo $n$.
\end{problem}

\begin{problem}
Prove your remainder rule works for finding remainders modulo $4$.  
\end{problem}

\begin{problem}
Prove your remainder rule works for finding remainders modulo $9$.  
\end{problem}

\begin{problem}
Prove Theorem 1 above.
\end{problem}

\begin{problem}
Prove Theorem 2 above. 
\end{problem}

\begin{problem}
Explain how Theorem 2 establishes that ``equivalence mod $n$'' is indeed an equivalence relation. 
\end{problem}

\begin{problem}
Explain how Theorem 2 establishes that the operations of addition, multiplication, and exponentiation are ``well defined.''
\end{problem}

\begin{problem}
Which numbers have multiplicative inverses mod $n$?  What does it mean to be a multiplicative inverse?  
\end{problem}


\end{document}
