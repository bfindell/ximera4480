%\documentclass{ximera}
\documentclass[space,handout,nooutcomes]{ximera}

\newcommand{\N}{\mathbb N}
\newcommand{\W}{\mathbb W}
\newcommand{\C}{\mathbb C}
\newcommand{\Z}{\mathbb Z}
\newcommand{\Q}{\mathbb Q}
\newcommand{\R}{\mathbb R}

\title{Sequences and Series}

\begin{document}
\begin{abstract}
Exploring sequences and series as functions.    
\end{abstract}
\maketitle


\begin{problem}
Suppose a sequence begins with 4 and 7 as the first and second terms. 
\begin{enumerate}
\item Supposing that it is an arithmetic sequence, write the next three terms. 
\item Supposing that it is an arithmetic sequence, write a formula for the $n^\textrm{th}$ term.  
\item Supposing that it is a geometric sequence, write the next three terms. 
\item Supposing that it is a geometric sequence, write a formula for the $n^\textrm{th}$ term.
\item What is the same and what is different about your answers to parts (b) and (d).     
\item Supposing that the sequence is neither arithmetic nor geometric, write three more terms.  
\end{enumerate}
\end{problem}

\begin{problem}
Consider the arithmetic series
\[
7 + 10 + 13 + \dots + 405
\]
\begin{enumerate}
\item Find the sum of the series.  
\item Write the sequence of partial sums of the series.  
\item Find the sum of the first $n$ terms of the series, assuming that the first term is 7. 
\item What kind of function is your answer to part (b).  Why does this make sense? 
\end{enumerate}
\end{problem}


\begin{problem}
Consider the general geometric series
\[
a + ar + ar^2 + \dots + ar^n
\]
Derive a formula for the sum of this series.  
\end{problem}

\begin{problem}
Consider the repeating decimal $0.\overline{23} = 0.232323\dots$.  
\begin{enumerate}
\item Express this repeating decimal as a fraction in lowest terms. 
\item Express this repeating decimal as a geometric series. 
\item Find a general expression for the sum of the first $n$ terms of this geometric series. 
\item Explain what happens to the sum of the finite series as the number of terms tends to infinity.   
\end{enumerate}
\end{problem}

\begin{problem}
Express the repeating decimal $4.5\overline{237} = 4.5237237237\dots$ as a fraction in lowest terms.  Explain your reasoning.   
\end{problem}

\begin{problem}
Is $0.\overline{9}=1$?  Provide two explanations for your answer.  
\end{problem}

\begin{problem}
Explain briefly why a real number is rational if and only if its decimal representation is either terminating or repeating.  
\end{problem}

\end{document}

